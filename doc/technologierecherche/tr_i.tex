\section{Technologierecherche Informatik}
Um eine umfangreiche Recherche im Bereich Informatik sicherzustellen, wurde 
sie in vier Bereiche gegliedert: Positionserkennung des Korbes, 
Schnittstellen, Bilderkennung / –verarbeitung und Positionsberechnung. Im 
Bereich Positionserkennung des Korbes wurden verschiedene Kameratechnologien 
angeschaut. Im Bereich Schnittstellen wurden heute gängige Standards zur 
Datenübertragung angeschaut. In der Bilderkennung / –verarbeitung wurden 
bestehende Technologien gesucht, die zur Bilderkennung / -verarbeitung 
verwendet werden können. Das Thema Positionsberechnung setzt sich damit 
auseinander, wie die Position des Fahr- bzw. Flugzeugs auf dem Spielfeld 
erkannt werden kann. Nachfolgend sind diese Links in Tabellenform ersichtlich:

\subsection{Positionserkennung Korb HW}

\subsubsection{Raspberry Pi – Kamera Module}
\url{http://www.raspberrypi.org/camera} \\
\begin{itemize}
    \item Ab 35.- erhältlich
    \item Direkter Anschluss an RaspberryPi
    \item 2592 x 1944 Pixel Auflösung.
\end{itemize}

\subsubsection{Herkömmliche Webcam}
\url{http://www.toppreise.ch/prod_181564.html}
\begin{itemize}
    \item Ab 13.- erhältlich
    \item Leistungsstärkere Modelle verfügbar
    \item zB. Microsoft LifeCam VX 800 640 x 480 px
\end{itemize}

\subsubsection{Smartphonekamera}
\url{http://de.wikipedia.org/wiki/Smartphone}
\begin{itemize}
    \item ältere Smartphones billig verfügbar
    \item gute Kamera, inkl. Prozessoren und Memory
    \item viele API’s
    \item integriertes WLAN und Webservices
\end{itemize}

\subsection{Schnittstellen}

\subsubsection{USB}
\url{http://de.wikipedia.org/wiki/Raspberry_Pi}
\begin{itemize}
    \item Microcontroller bieten meist eine USB- oder Mikro-USB-Schnittstelle
    \item Smartphones sind meist über eine Mikro-USB-Schnittstelle verfügbar
\end{itemize}

\subsubsection{Bluetooth}
\url{http://www.raspberrypi.org/forums/viewtopic.php?f=45&t=27678}
\begin{itemize}
    \item Smartphones sind meist Bluetooth fähig
    \item Microcontroller können mit einem zusätzlichen Bluetooth-Sender erweitert werden
    \item RaspberryPi verfügt über einen Bluetooth-Dongle
    \item Es sind keine Kabel oder statische Verbindungen notwendig
\end{itemize}

\subsubsection{GigaInfrarot}
\url{http://computer-oiger.de/2012/10/28/drahtlos-mit-tempo-3-gigabit-dresdner-forscher-entwickeln-neue-optik-schnittstelle/17936}
\begin{itemize}
    \item Neue Technologie
    \item schnellen Datenübertragung
    \item Jedoch nur auf kurzer Distanz möglich.
\end{itemize}

\subsubsection{Infrarot}
\url{http://www.itwissen.info/uebersicht/lexikon/Infrarot-LANs.html?page=0}
\begin{itemize}
    \item Unterschiedliche Protokolle
    \item flexibler
\end{itemize}

\subsubsection{WifiDirect}
\url{http://www.wi-fi.org/discover-wi-fi/wi-fi-direct}
\begin{itemize}
    \item Punkt-zu-Punkt-Verbindung
    \item Website mit Beschreibung
\end{itemize}

\subsection{Bilderkennung und Verarbeitung}

\subsubsection{Bildverarbeitung}
\url{http://www.kreissl.info/bilderkennung.php}
\begin{itemize}
    \item Allgemeines Vorgehen bei der Bildverarbeitung. 
    \begin{itemize}
        \item Vorgehen
        \item Segmentierung
        \item Verarbeitung
    \end{itemize}
    \item Zusätzliche Infos anhand von einem Beispielbild.
\end{itemize}

\subsubsection{OpenCV}
\url{http://opencv.org/ }
\begin{itemize}
    \item Enormer funktionsumfang
    \item Grosse Community
    \item Alle gängigen Plattformer werden unterstützt
    \item Plugins für IDE’s oder compability-packs verfügbar
\end{itemize}

\subsubsection{LabView}
\url{http://www.ni.com/trylabview/d/?metc=mtnpee&gclid=CIn79ZGL67kCFUlZ3godWzYAgQ}
\begin{itemize}
    \item Speziell für Bildbearbeitung
    \item Grosse Treiberbibliotheken
    \item Grosse Community
    \item Nur Windows
\end{itemize}

\subsubsection{Matlab}
\url{http://www.mathworks.ch/products/matlab/}
\begin{itemize}
    \item Windows, Linux und Mac
    \item Java, C, C++ oder .NET
    \item Evtl. keine Kameraunterstützung
\end{itemize}

\subsection{Positionsberechnung}

\subsubsection{Mikrokontroller}
\url{http://de.wikipedia.org/wiki/Raspberry_Pi}
\begin{itemize}
    \item Direkte Berechnung auf Microcontroller, deshalb keine Übertragung notwendig
    \item Hohe Frequenzen führen eventuell zu Überlastung vom Microcontroller
    \item Hat andere Aufgaben (Motorensteuerung)
\end{itemize}

\subsubsection{Smartphones}
\url{http://de.wikipedia.org/wiki/Smartphone}
\begin{itemize}
    \item Wenn Bild auf Smartphone, gleich auch Bildauswertung
    \item Kompatibilität mit der Bildverarbeitungs-Software muss gegeben sein
    \item Kompatibilität mit dem Microcontroller
\end{itemize}

\subsubsection{Tablet / PC}
\url{http://de.wikipedia.org/wiki/Notebook}
\begin{itemize}
    \item Grosse Leistung
    \item Übertragung notwendig
    \item Internetnutzung möglich
    \item Eventuell Bildverarbeitung auf einem externen Server
\end{itemize}

\subsubsection{Geometrischer Schwerpunkt}
\url{http://www.gerdlamprecht.de/GeometrischerSchwerpunkt.htm}
\begin{itemize}
    \item Einfach umsetzbar
    \item Einfach berechenbar
    \item Evtl. auf Microcontroller
\end{itemize}

\subsubsection{Entfernungsmessung}
\url{http://de.wikipedia.org/wiki/Entfernungsmessung}
\begin{itemize}
    \item Unterschiedliche Messtechniken (siehe Link)
\end{itemize}
