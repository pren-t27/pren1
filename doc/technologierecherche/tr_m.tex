\section{Technologierecherche Maschinenbau}

\subsection{Einleitung}
Die Technologierecherche im Bereich Maschinenbau gliedert sich in die zwei 
Bereiche, „Fortbewegung“ und  „Beförderung der Bälle“. Falls der Entscheid auf ein Flugobjekt fallen würde, würden sich diese Bereiche überschneiden und es müsste noch eine Technologie entwickelt werden um die Bälle vom Flugobjekt aus zu klinken. Um sich nicht zu früh fest zu fahren wurden bewusst auch Technologien recherchiert die vielleicht auf dem ersten Blick nicht sehr sinnvoll erscheinen.

\subsection{Fortbewegung}

\subsubsection{Raupen \hfill Bewertung: 7}
\url{http://www.dawnrobotics.co.uk/rover-5-seeeduino-arduino-robot-kit/}
Vor- / Nachteile:
\begin{itemize}
    \item[+] keine Steuerbaren Achsen zum Kurvenfahren nötig 
    \item[+] Richtungsteuerung nur über Drehzahl möglich
    \item[+] müsste nur in eine Richtung fahren (seitlich)
    \item[-] evtl. langsam
    \item[-] kompliziertere Mechanik
\end{itemize}

\subsubsection{Räder \hfill Bewertung: 9}
Vor- / Nachteile:
\begin{itemize}
    \item[+] einfache Konstruktion und Mechanik
    \item[+] weniger Bewegte Masse
    \item[+] müsste nur in eine Richtung fahren (seitlich)
    \item[+] viele Standardteile aus dem Modellbausektor vorhanden
    \item[-] zum Kurvenfahren aufwändige Mechanik nötig
\end{itemize}

\subsubsection{Quadrocopter \hfill Bewertung: 7}
\url{http://de.wikipedia.org/wiki/Quadrocopter} \\
Vor- / Nachteile:
\begin{itemize}
    \item[+] schnell (kein Nachladen nötig)
    \item[+] präzise
    \item[+] geringes Gewicht
    \item[+] weniger mechanische Elemente (Bälle müssen nicht geworfen werden)
    \item[-] Komplizierte Steuerung da Stabilisierung des Quadcopters selber 
        gemacht werden müsste
    \item[-] System um die Bälle in den Korb herunter zu lassen
    \item[-] Eine Dimension mehr, da Höhe gemessen und gesteuert werden muss
    \item[-] mechanische Stabilisierung fast nicht möglich
\end{itemize}

\subsection{Beförderung der Bälle}

\subsubsection{Druckluft \hfill Bewertung: 3}
\url{http://www.youtube.com/watch?v=yl_hdBXrVXk} \\
Eigenbau. Bälle werden mit Laubgebläse beschleunigt. \\
\url{http://www.youtube.com/watch?v=M5_xbuWW7Qc} \\
Eigenbau. Bälle werden mittels Druckluft aus Tank beschleunigt. \\
Vor- / Nachteile:
\begin{itemize}
    \item[+] einfache Mechanik
    \item[+] keine beweglichen Teile
    \item[-] Druckluftbehälter
    \item[-] Druckmessung um konstanten Druck zu haben
    \item[-] hohes Gewicht
    \item[-] ungenau durch Grössenunterschiede der Bälle
\end{itemize}

\subsubsection{Ballwurfmaschine mit zwei Rad \hfill Bewertung: 8}
\url{http://www.youtube.com/watch?v=oZjx7F1doGs} \\
Eigenbau. Bälle werden zwischen 2 drehenden Rädern beschleunigt. Spin ist über 
Drehzahl der Räder einstellbar. \\
\url{https://www.youtube.com/watch?v=YubwWqucVcI} \\
Positionierungsmöglichkeiten über Mikrocontroller / Schrittmotoren. Schnelle 
Schussfolge. Drehzahl einstellbar durch Mikrocontroller \\
\url{http://www.tennis-aaron.de/tennisballmaschinen.htm} \\
Professionelle Tennisballwurfmaschine. \\
Vor- / Nachteile:
\begin{itemize}
    \item[+] gut Steuerbar
    \item[+] konstant
    \item[+] einfache Mechanik
    \item[+] kein Drall
    \item[-] Schwungrad nötig (grösseres Gewicht)
\end{itemize}

\subsubsection{Ballwurfmaschine mit einem Rad \hfill Bewertung: 9}
\url{http://jugssports.com/pitching-machines/softball-pitching-machine/} \\
Professionelle Pitchingmachine. \\
Vor- / Nachteile:
\begin{itemize}
    \item[+] gut Steuerbar
    \item[+] konstant
    \item[+] einfachste Mechanik
    \item[+] leichter da nur ein Rad
    \item[-] Drall
    \item[-] Schwungrad nötig
\end{itemize}

\subsubsection{Feder \hfill Bewertung: 8}
\url{http://www.airsoftct.com/electric-airsoft-gun-gearbox-guide/} \\
Funktionsweise Airsoftgewehr \\
\url{http://www.youtube.com/watch?v=ZuJlHWvjlno} \\
Vorhandene Ballwurfmaschine für Hunde. Keine Führung bei Wurfvorgang. Relativ günstig. \\
\url{http://www.sirblackhand.com/pages/crossbows.html} \\
Armbrust \\
Vor- / Nachteile:
\begin{itemize}
    \item[+] sehr konstante Beschleunigung
    \item[+] leicht
    \item[-] Rückschlag
    \item[-] komplizierte Mechanik
    \item[-] komplizierte Justierung
\end{itemize}

\subsubsection{Trebuchet \hfill Bewertung: 1}
\url{http://www.youtube.com/watch?v=8hAX72Xgf1U} \\
Physikalische Beschreibung eines Trebuchet \\
Vor- / Nachteile:
\begin{itemize}
    \item[+] keine
    \item[-] hohes Gewicht
    \item[-] komplizierter und ungenauer Aufbau
\end{itemize}

\subsubsection{Katapult \hfill Bewertung: 3}
\url{http://dmrc.uni-paderborn.de/uploads/media/Katapulte_Industrieanzeiger_31_2011.pdf} \\
Vor- / Nachteile:
\begin{itemize}
    \item[+] konstant
    \item[-] hohes Gewicht
    \item[-] grosse bewegte Masse
    \item[-] grosse Dimensionen
\end{itemize}

\subsection{Technologiebücher}
Leichtbau, Elemente und Konstruktion, Johannes Wiedemann, 3. Auflage, 
978-3-540-33656-3 \\
Leichtbau Prinzipien, Werkstoffauswahl und Fertigungsvarianten, Hans Peter Degischer und Sigrid Lüftl, 978-3-527-32372-2
