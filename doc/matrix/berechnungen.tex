\section{Anhang: Berechnungen}

\subsection{Spielfeld}
Angenommen das Gerät wird im Startfeld mittig an der Startlinie positioniert, und ausschliesslich mit einer drehbaren Abschussvorrichtung ausgerüstet, und der Korb wird ganz am Rand des Feldes positioniert, ergibt sich für die maximale Wurflänge lmax folgender Wert: 
 
\[\ l_\text{max} = \sqrt{(600mm)^2 + (1900mm)^2} = 1992.49mm \]

Somit ergibt sich im schlechtesten Fall ein Längenunterschied von ca 9.3 cm. Je nachdem wie hoch die Wurfgenauigkeit ist, könnte man die 9.3cm in Anbetracht des Korbdurchmessers von 30cm fast vernachlässigen, und ein Drehturm mit immer gleicher Wurflänge konstruieren. Voraussetzung ist, dass eine gute Zielgenauigkeit erreicht werden kann.

\subsection{Ballwurf}
Angenommen der Ball wird auf der Höhe des Korbes abgeworfen, kann mit 
folgender Formel berechnet werden, mit welcher Geschwindigkeit der Ball 
geworfen werden muss, falls er in einem Winkel von 65º geworfen wird:
%
\[ v_0 
= \sqrt{ \frac{s_x}{\sin(2\alpha)} \cdot g } 
= \sqrt{ \frac{1.9m}{\sin(2 \cdot 65^\circ)} \cdot 9.81 \frac{m}{s^2}} 
= 4.93 \frac{m}{s} \]
%
Die maximale Höhe von 1.8 Metern darf nicht überschritten werden. Die Höhe die 
der Ball erreicht lässt sich folgendermassen berechnen:
%
\[ h_\text{max} 
= \frac{v_0^2 \cdot \sin(\alpha)^2}{2 \cdot g} 
= \frac{4.93 \frac{m}{s}^2 \cdot \sin(65^\circ)^2}{2 \cdot 9.81 \frac{m}{s^2}} 
= 1.018m \]
%
Die Zeit, die der Ball braucht bis er im Korb ist, lässt sich folgendermassen 
berechnen:
%
\[ t = \frac{s_x}{v_0 \cdot \cos(\alpha)} 
= \frac{1.9m}{4.93 \frac{m}{s} \cdot \cos(65^\circ)} = 0.911s \]
%
Nun werden die gleichen Rechnungen mit einem Winkel von 70º durchgeführt:
%
\[ v_0 = \sqrt{ \frac{s_x}{\sin(2\alpha)} \cdot g } 
= \sqrt{ \frac{1.9m}{\sin(2 \cdot 70^\circ)} \cdot 9.81 \frac{m}{s^2}} 
= 5.385 \frac{m}{s} \]

\[ h_\text{max} = \frac{v_0^2 \cdot \sin(\alpha)^2}{2 \cdot g} 
= \frac{5.385 \frac{m}{s}^2 \cdot \sin(70^\circ)^2}{2 \cdot 9.81 \frac{m}{s^2}} 
= 1.305m \]

\[ t = \frac{s_x}{v_0 \cdot \cos(\alpha)} 
= \frac{1.9m}{5.385 \frac{m}{s} \cdot \cos(70^\circ)} = 1.032s \]
%
Geht man nun beispielsweise davon aus, dass die Abschussgenauigkeit ± 0.3 m/s 
beträgt, kann berechnet werden, mit welchem Winkel die Distanz mehr variiert:

\noindent
Winkel 65º:
%
\[ s_x = \frac{(v_0 \pm \Delta v)^2 \cdot \sin(2 \cdot \alpha)}{g} 
= \frac{(4.93 \frac{m}{s} \pm 0.3 \frac{m}{s})^2 \cdot \sin(2 \cdot 65^\circ)}{9.81 \frac{m}{s^2}} \]

\[ s_\text{xmax} = 2.136m \hspace{30mm} s_\text{xmin} = 1.674m \]
%
Winkel 70º:
%
\[ s_x = \frac{(v_0 \pm \Delta v)^2 \cdot \sin(2 \cdot \alpha)}{g} 
= \frac{(5.385 \frac{m}{s} \pm 0.3 \frac{m}{s})^2 \cdot \sin(2 \cdot 70^\circ)}{9.81 \frac{m}{s^2}} \]

\[ s_\text{xmax} = 2.118m \hspace{30mm} s_\text{xmin} = 1.694m \]
%
Bei einer Abschussgeschwindigkeitsänderung von lediglich ± 0.3 m/s ergibt sich 
bereits eine Abweichung der Wurfweite von ± 20-25cm. Die Geschwindigkeit muss 
somit ziemlich genau eingestellt werden können und sehr konstant sein.

\noindent
Auch kann man nun sagen, dass die Längenänderung mit grösser werdendem Winkel 
kleiner wird. Somit wäre es besser, den Ball in einem grösseren Winkel zu 
werfen. Allerdings muss dann der Ball schneller geschossen werden (mehr 
Energieaufwand, grösserer Rückstoss), der Weg wird länger uns der Flug dauert 
länger.
