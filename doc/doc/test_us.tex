\subsection{Ultraschallsensor HC-SR04}
Diese Messung wird gemeinsam mit Gruppe 39 durchgeführt. \\
Ein sehr verbreitetes Modul für die Distanzmessung mittels Ultraschall ist 
das Modul HC-SR04. Mit diesem werden Messungen durchgeführt um festzustellen, 
ob es geeignet ist, einen Eimer zu detektieren. 

\subsubsection{Messmittel}
\begin{zebratabular}{lll}
    \rowcolor{gray} Gerät &
        Typ &
        Nummer \\
    Speisegerät &
        Hameg ... &
        SN ... \\
    Oszilloskop &
        Agilent &
        SN ... \\
    Pulsgenerator &
        Hameg &
        SN ... \\
    Mainframe &
        Hameg 800? &
        SN ... \\
\end{zebratabular} \\
Die Messungen werden im Raum B332c durchgeführt. 

\subsubsection{Ansteuerung}
\begin{zebratabular}{ll}
    \rowcolor{gray} Pin & Beschreibung \\
    VCC     & +5V DC \\
    Trig    & Trigger-Eingang (Startsignal) \\
    Echo    & Echo-Feedback \\
    GND     & Masse (0V) \\
\end{zebratabular}

\subsubsection{Testeimer}
Als Testeimer wird der Abfalleimer aus dem Raum C200 verwendet. \\
\begin{zebratabular}{ll}
    \rowcolor{gray} Eigenschaft & Wert \\
    Durchmesser oben    & 38 cm \\
    Durchmesser unten   & 33 cm \\
    Höhe                & 48 cm \\
    Farbe               & Schwarz (matt) \\
    Material            & Kunststoff () \\
    Hersteller          & Helit \\
    Typ                 & 61062 \\
\end{zebratabular}

\subsubsection{Messung Messgenauigkeit}
Die folgenden Werte sind statistisch aus mindestens 1000 Einzelmessungen 
ermittelt. \\
\begin{zebratabular}{lll}
    \rowcolor{gray} Abstand [cm] & Impuls mean [ms] & Std. Dev. [$\mu$s] \\
    50  & 2.987 & 2.4 \\
    60  & 3.503 & 2.4 \\
    70  & 4.060 & 10 \\
    80  & 4.766 & 24 \\
    90  & 5.230 & 10 \\
    100 & 5.807 & 11 \\
    110 & 6.413 & 13 \\
    120 & 7.040 & 16 \\
    130 & 7.722 & 22 \\
    140 & 8.229 & 16 \\
    150 & 8.854 & 15 \\
    160 & 9.500 & 43 \\
    170 & 10.06 & 22 \\
    180 & n.a.  & n.a. \\
\end{zebratabular} \\
Anschliessend wird eine weitere Messung durchgeführt. Dabei wird eine 
Holzplatte mit einem Abstand von 180 cm verwendet. Als Platte dient ein Tablar 
aus dem Raum B332c. Der Median beträgt 10.45 ms bei einer Standardabweichung 
von 9.7 $\mu$s. \\
$\to$ Deutlich besseres Signal auf flachen Gegenständen als auf Runden. 

\subsubsection{Messung seitliche Empfindlichkeit}
Um die seitliche Empfindlichkeit zu testen, wird der Eimer unter einem 
bestimmten Winkel vor dem Sensor aufgestellt. Der Abstand wird dabei 
so eingestellt, dass der Sensor den Eimer gerade noch erkennt. Die erzielte
Distanz wird gemessen. \\
\begin{zebratabular}{ll}
    Winkel [$^\circ$] & Messbereich [cm] \\
    0   & 180 \\
    5   & 123 \\
    10  & 120 \\
    15  & 119 \\
    20  & 113 \\
    25  & 106 \\
    30  & 104 \\
    35  & 77  \\
    40  & 0   \\
\end{zebratabular} \\
Zwischen 25$^\circ$ und 30$^\circ$ hat der Sensor in einem Abstand von 75 bis 
90 cm einen blinden Bereich, in welchem der Eimer nicht erkannt wird. 

\subsubsection{Fazit}
Der Sensor ist nicht geeignet, um den Eimer zu finden, da der Sensor über 
einen zu breiten und zu ungenau definierten Winkel empfindlich ist. Der 
HC-SR04 wird somit nicht weiter als Sensor zur Detektion des Eimers 
weiterverfolgt. 

