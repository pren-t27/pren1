\section{Tests und Versuche}

\subsection{Mechanik}


Die Hauptaufmerksamkeit zu Beginn galt dem Prozess des Schiessens/ Befördern der Bälle. Aus diesem Grund wurde entschieden, einen Versuchsaufbau zu konstruieren, um diese Funktion auf ihre Zuverlässigkeit und Genauigkeit zu testen. Daher wurde ein Aufbau hergestellt, welcher die Bälle mit Hilfe eines Rades beschleunigt.
%
%
\\Die Konstruktion bestand hauptsächlich aus Holz, damit relativ einfach Anpassungen vorgenommen werden können. Die Lagerung der Welle, mit welcher das Rad aus dem Modellbau drehte, wurde mit zwei im Holz eingepressten Kugellager realisiert. Da die berechnete Drehzahl des Rades zwischen 700 U/min bis 900 U/min lag, musste ein passender Motor gefunden werden. Motoren in diesem Drehzahlbereich sind jedoch sehr rar. Aus diesem Grund wurde auf einen Motor welcher bei RC-Modellautos eingesetzt wird zurückgegriffen. Dieser hat jedoch eine Drehzahl von über 20'000 U/min und musste deshalb stark untersetzt werden, um die gewünschte Raddrehzahl zu erhalten. Die Untersetzung wurde mit zwei Riemenscheiben und einem O-Ring gelöst. Dadurch wurde eine Untersetzung von ca. 1:30 erreicht damit der Ball die richtige Abschussgeschwindigkeit erhält. Da dieser Motor leider einen relativ hohen Drehzahl Einbruch hatte sobald ein Ball geschossen wurde, musste nach einer Alternative gesucht werden. Der zweite Motor mit welchem getestet wurde, stammt aus dem Flugzeugmodellbau und war ein brushless aussenläufer Motor mit Getriebe. Mit diesem konnten die Bälle in sehr kurzen Abständen mit nahezu gleich bleibender Drehzahl geschossen werden.Der Abschusswinkel bei diesen Versuchen wurde experimentell ermittelt und mit ca. 50° als optimal festgelegt. 
%
%
\\Anschliessend wurde ein Mechanismus gesucht mit welchem die Bälle gleichmässig zum Beschleunigungsrad geführt werden, damit die Wurfweite nicht beeinflusst wird. Die Idee war es, mit einem umschlingenden Band die Bälle nacheinander zuzuführen. Beim Versuch wurde als Band eine 0.1 mm dicke Präzisionsstahlfolie verwendet. Diese zeichnet sich durch die hohe Stabilität, Aufrollbarkeit und das geringe Gewicht aus. Beim Test war das Band auf der einen Seite befestigt, umschlingte alle Tennisbälle und wurde kurz vor dem Rad durch einen Schlitz im Holz nach draussen geführt. Leider konnte bis heute diese Technik nur manuell getestet werden, indem das Band beim Test von Hand hinausgezogen wurde.\\
%
Den Test des Drehmechanismus konnte leider noch nicht durchgeführt werden, da einzelne Komponenten noch fehlen und dadurch diese noch nicht kombiniert werden können.




\subsection{Test 2}

\subsection{Test 3}