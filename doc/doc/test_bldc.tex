\subsection{BLDC Ansteuerung}
Diese Messung wird gemeinsam mit Gruppe 32 durchgeführt. \\

\subsubsection{Aufbaubeschreibung}
Dieses Kapitel ist eine Zusammenarbeit der Gruppen T27 und T32. 
    Das Schema des Gesamtaufbaus des Tests ist in der Abbildung 
    \ref{abb:MotoransteuerungSchema} ersichtlich. Die 3-Phasen H-Brücke 
    im oberen grünen Rechteck wird direkt vom FPGA angesteuert. Die Hardware 
    dieser Brücke ermöglicht eine voll galvanisch getrennte Ansteuerung 
    mit 3.3V Logikpegeln. Diese Brücke wurde zur Verfügung gestellt und direkt
    implementiert. Die Rekonstruktion der Hallsensoren-Signale findet im rot 
    markierten Teil des Aufbaus statt. Dieser Part wurde auf einer 
    Laborplatte aufgebaut und gelötet. Die so generierten Signale 
    $U_{Hallsensor}$, $V_{Hallsensor}$, $W_{Hallsensor}$ werden einem FPGA 
    geliefert. Anhand dieser Signale steuert das FPGA die 
    H-Brücken-Transistoren mit den Signalen $U_h$, $U_l$, $V_h$, $V_l$, 
    $W_h$, $W_l$. Die im FPGA enthaltene Konfiguration besteht aus simplen 
    AND-Verknüpfungen, die die anliegenden Signale sehr schnell und 
    effizient verarbeiten können. Auf diese Weise ist es möglich, den Motor sehr 
    schnell anzusteuern.
    \begin{figure}[h!]
    	\includegraphics[scale=0.4]{fig/MotoransteuerungSchema.jpg}
       	\centering
       	\caption{Schema des Brushless-Versuchsaufbaus}
        \label{abb:MotoransteuerungSchema}
    \end{figure}
    In der Abbildung \ref{abb:MessplatzAufbau} ist der gesamte Aufbau 
    abgebildet. Man beachte die markierten Felder. Am linken unteren Rand 
    ist der Motor befestigt. In der Mitte des Bildes ist die Hardware zur Rekonstruktion der Hallsensoren-Signale.
    Die generierten Signale werden dem FPGA in der unteren linken Ecke zugeführt. Diese 
    Signale werden logisch verknüpft und danach die sechs Signale 
    generiert, um die H-Brücke in der rechten oberen Hälfte anzusteuern. 
    Die H-Brücken wiederum treiben den Motor an.
    \begin{figure}[h!]
    %\vspace{-16pt}
       	\includegraphics[scale=0.14]{fig/MessplatzAufbau.jpg}
       	\centering
       	\caption{Testaufbau} 
        \label{abb:MessplatzAufbau}
    %\vspace{-10pt}
    \end{figure}
    Die im FPGA enthaltene Logik basiert auf der Wahrheitstabelle, die in 
    Abbildung \ref{abb:WahrheitstabelleAnsteuerung} abgebildet ist.

\subsubsection{Messmittel}
    \begin{table}[h!]
        \centering
        \begin{zebratabular}{lll}
            \rowcolor{gray}
            Gerät &
                Typ &
                Nummer \\
            Speisegerät & 
                Rohde \& Schwarz NGSM 32/10 &
                Inv.-Nr. 009 \\
            Oszilloskop &
                Agilent MSO6052A &
                Inv.-Nr. 44; S/N: MY44001903 \\
            Mainframe &
                Hameg HM8001-2 &
                SN: 059520046 \\
            Speisegerät &
                Hameg HM8040-3 &
                SN: 015405014 \\
            Pulsgenerator &
                Hameg HM8035 &
                Inv.-Nr. 44 \\
        \end{zebratabular}
        \caption{Messmittel des Versuchsaufbaus}
    \end{table}
