\pagestyle{empty}
\begin{landscape}
    \section{Risikoanalyse}
	\begin{footnotesize}
		\subsection{Risikoanalye allgemein}
		\begin{table}[h!]
			\begin{zebratabular}{@{}clllp{0.25\linewidth}}		
				\textbf{ID}&\textbf{Beschreibung}&\textbf{Wahrscheinlichkeit}&\textbf{Auswirkung}&\textbf{Massnahmen}\\
				\hline
				\#1&Teammitglied Elektrotechnik fällt aus&sehr niedrig&sehr hoch&keine\\
				\#2&Teammitglied Maschinenbau fällt aus&sehr niedrig&hoch&Übernahme durch Maschinenbauer\\
				\#3&Teammitglied Informatik fällt aus&sehr niedrig&hoch&Übernahme durch Informatiker\\
				\#4&Budget wird überschritten&niedrig&hoch&Vorab Kosten abklären\\
				\#5&Zeit reicht nicht zur Realisierung&mittel&sehr hoch&Zeitplanung mit Meilensteinen\\
				\#6&Eingekaufte Komponente fällt aus&niedrig&hoch&Neu bestellen\\
				\#7&Komponente (Eigenbau) fällt aus&niedrig&mittel&Neu bauen\\
				\#8&Komponente funktioniert nicht&niedrig&mittel&Komponente reparieren\\
				\#9&Berechnungen falsch&sehr niedrig&hoch&Überprüfung durch mehrere Personen\\
				\#10&Abmessungseinschränkungen überschritten&sehr niedrig&hoch&Modell bauen\\
				\#11&Schnittstellen ungenau definiert&sehr niedrig&sehr hoch&Überprüfung durch mehrere Personen\\
				\#12&Startsignal funktioniert nicht&niedrig&hoch&Ausgiebiges Testen\\
				\#13&Endsignal nicht übermittelt&niedrig&sehr niedrig&Ausgiebiges Testen\\
				\#14&Verfehlen des Korbs&mittel&mittel&Ausgiebiges Testen\\
				\#15&Stromversorgung reicht nicht aus&sehr niedrig&mittel&Ausgiebiges Testen\\
				\#16&Anforderungen ändern&sehr niedrig&hoch&Neu planen\\
			\end{zebratabular}
		\end{table}
        \clearpage
		\subsection{Risikoanalye Konzept 1 (Flugobjekt)}
		\begin{table}[h!]
			\begin{zebratabular}{@{}clllp{0.25\linewidth}}		
				\textbf{ID}&\textbf{Beschreibung}&\textbf{Wahrscheinlichkeit}&\textbf{Auswirkung}&\textbf{Massnahmen}\\
				\hline
				\#17&Seitliches Abdriften&hoch&hoch&Stabilisation einbauen\\
				\#18&Unkontrollierte Steigung&niedrig&sehr hoch&Notschalter einbauen\\
				\#19&Ausfall eines Motors&niedrig&hoch&keine\\
				\#20&Gewicht zu gross (hebt nicht ab)&sehr niedrig&sehr hoch&Ausgiebige Planung\\
				\#21&Ausklinken funktioniert nicht&sehr niedrig&mittel&Ausgiebiges Testen\\
			\end{zebratabular}
		\end{table}
		\subsection{Risikoanalye Konzept 2 (Bodenobjekt stehend)}
		\begin{table}[h!]
			\begin{zebratabular}{@{}clllp{0.25\linewidth}}		
				\textbf{ID}&\textbf{Beschreibung}&\textbf{Wahrscheinlichkeit}&\textbf{Auswirkung}&\textbf{Massnahmen}\\
				\hline
				\#22&Drehung funktioniert nicht&sehr niedrig&mittel&Ausgiebiges Testen\\
				\#23&Abschussmechanismus fällt aus&sehr niedrig&hoch&Ausgiebiges Testen\\
				\#24&Balllager (Nachschieben) streikt&sehr niedrig&hoch&Mechanismus einbauen, der nicht blockiert\\
				\#25&Umfallen durch Rückstoss&sehr niedrig&niedrig&Ausgiebiges Testen\\
				\#26&Genauigkeit des Schusses&mittel&hoch&Gleichmässige Beschleunigung garantieren\\
			\end{zebratabular}
		\end{table}
        \clearpage
		\subsection{Risikoanalye Konzept 3 (Bodenobjekt fahrend)}
		\begin{table}[h!]
			\begin{zebratabular}{@{}clllp{0.25\linewidth}}		
				\textbf{ID}&\textbf{Beschreibung}&\textbf{Wahrscheinlichkeit}&\textbf{Auswirkung}&\textbf{Massnahmen}\\
				\hline
				\#27&Fahrmechanismus funktioniert nicht&niedrig&hoch&Ausgiebiges Testen\\
				\#28&Ausrichten funktioniert nicht&niedrig&sehr hoch&Ausgiebiges Testen\\
				\#29&Ballager (Nachschieben) streikt&sehr niedrig&hoch&Mechanismus einbauen, der nicht blockiert\\
				\#30&Umfallen durch Rückstoss&sehr niedrig&niedrig&Ausgiebiges Testen\\		
			\end{zebratabular}
		\end{table}
	\end{footnotesize}
\end{landscape}
