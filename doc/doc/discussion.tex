\section{Schlussdiskussion}

\subsection{Erfahrungen}

\subsubsection*{Daniel Mathis}
Zu Beginn war es wichtig, die Teammitglieder kennenzulernen, um so ein 
möglichst breites Vorwissen aufzubauen. Anhand dieses Vorwissens, 
konzentrierten wir uns schnell auf eine Lösung welche fliegen sollte. Die uns 
anschliessend in Auftrag gegebene Risikoanalyse zeigte jedoch auf, dass diese 
Lösung ein zu hohes Risiko darstellt. Dieses zeigte mir persönlich die 
Wichtigkeit eines strukturierten Ablaufes auf. Des weiteren war es nicht ganz 
einfach, die Aufgabenstellung zu unterteilen in die verschiedenen Teilsysteme 
und zu diesen Teilsystemen möglichst viele Möglichkeiten aufzuzeigen. 
Rückblickend betrachtend, gelang dies uns jedoch relativ gut. Eine weitere 
Schwierigkeit für mich waren die Berichte, welche abgegeben werden mussten. 
Deren Inhalt aus meiner Sicht nicht immer klar definiert war und uns daher 
etwas aus dem Zeitplan brachte. Im Grossen und Ganzen konnte ich jedoch viel 
von der Interdisziplinarität dieses Projektes profitieren und habe deshalb 
auch einiges über die Schnittstellen zwischen Mechanik und der Elektronik 
erlernt. Ich bin gespannt auf das nächste Semester, wenn wir unsere Ideen zu 
einer kompletten Maschine verbinden können.

\subsection{Offene Punkte/Risikien/Ausblick}
