\section{Schlussdiskussion}

\subsection{Erfahrungen}

Zu Beginn war es wichtig, die Teammitglieder kennenzulernen, um so ein 
möglichst breites Vorwissen aufzubauen. Anhand dieses Vorwissens, 
konzentrierten wir uns schnell auf eine Lösung, welche fliegen sollte. Die uns 
anschliessend in Auftrag gegebene Risikoanalyse zeigte jedoch auf, dass diese 
Lösung ein zu hohes Risiko darstellt. Dies zeigte uns die 
Wichtigkeit eines strukturierten Ablaufes auf. Des weiteren war es nicht ganz 
einfach, die Aufgabenstellung in die verschiedenen Teilsysteme zu unterteilen  
und zu diesen möglichst viele Lösungsmöglichkeiten aufzuzeigen. 
Rückblickend betrachtet, gelang uns dies jedoch gut. Eine weitere 
Schwierigkeit waren die Berichte, welche abgegeben werden mussten. 
Deren Inhalt war aus unserer Sicht nicht immer klar definiert war und brachte uns daher 
etwas aus dem Zeitplan.
Ebenfalls finden wir es einerseits gut, dass wir mit dem Drehturm eine - mehr oder weniger - einfache Lösung gefunden haben, andererseits schade, weil die Abteilung Informatik hier wenig zu tun hat. Im Grossen und Ganzen konnten wir jedoch viel 
von der Interdisziplinarität dieses Projektes profitieren und haben deshalb 
auch einiges über die Schnittstellen zwischen Informatik, Mechanik und der Elektronik 
erlernt. Die Dynamik, die in der Gruppe herrschte, wurde von uns als äusserst kollegial und lösungsorientiert empfunden und so freuen wir uns auf das Modul PREN2 mit diesem Team.


\subsection{Offene Punkte/Risikien/Ausblick}
