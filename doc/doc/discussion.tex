\section{Schlussdiskussion}

\subsection{Erfahrungen}

Zu Beginn war es wichtig, die Teammitglieder kennenzulernen, um so ein 
möglichst breites Vorwissen aufzubauen. Anhand dieses Vorwissens, 
konzentrierten wir uns schnell auf eine Lösung, welche fliegen sollte. Die uns 
anschliessend in Auftrag gegebene Risikoanalyse zeigte jedoch auf, dass diese 
Lösung ein zu hohes Risiko darstellt. Dies zeigte uns die 
Wichtigkeit eines strukturierten Ablaufes auf. Des weiteren war es nicht ganz 
einfach, die Aufgabenstellung in die verschiedenen Teilsysteme zu unterteilen  
und zu diesen möglichst viele Lösungsmöglichkeiten aufzuzeigen. 
Rückblickend betrachtet, gelang uns dies jedoch gut. Eine weitere 
Schwierigkeit waren die Berichte, welche abgegeben werden mussten. 
Deren Inhalt aus unserer Sicht nicht immer klar definiert war und uns daher 
etwas aus dem Zeitplan brachte.
Ebenfalls finden wir es einerseits gut, dass wir mit dem Drehturm eine - mehr oder weniger - einfache Lösung gefunden haben, andererseits schade, weil die Abteilung Informatik hier wenig zu tun hat. Im Grossen und Ganzen konnten wir jedoch viel 
von der Interdisziplinarität dieses Projektes profitieren und haben deshalb 
auch einiges über die Schnittstellen zwischen Informatik, Mechanik und der Elektronik 
erlernt. Die Dynamik, die in der Gruppe herrschte, wurde von uns als äusserst kollegial und lösungsorientiert empfunden und so freuen wir uns auf das Modul PREN2 mit diesem Team.


\subsection{Offene Punkte/Risikien/Ausblick}

\subsubsection{Ausblick PREN 2}
Die anfängliche Euphorie und die damit verbundene Idee zu fliegen konnte mit den im PREN 1 getätigten Arbeiten mit gutem Gewissen begraben werden. Zu Beginn war es noch schwierig von unserem Traum loszulassen doch unser neues Konzept konnte sich immer mehr beweisen bis schlussendlich alle Gruppenmitglieder davon überzeugt waren.
Für die Teilfunktionsmuster wurden bereits einige CAD-Modelle erstellt. Diese können nach Anpassungen auch teilweise für PREN2 verwendet werden. Die Informatiker haben im Zusammenhang mit der Technologierecherche ebenfalls bereits einige Codezeilen geschrieben. Diese können unter Umständen auch weiterverwendet werden.
Nach der bereits für PREN 1 geleisteten Arbeit schaut das ganze Team optimistisch auf PREN 2. Die Fertigstellung des Produktes gilt klar als die Hauptaufgabe für PREN 2. Um dies zu erreichen, ist eine reibungslose Kommunikation unter den Teilkomponenten essentiell. Ein Augenmerk gilt auch der Bauweise, die möglichst leicht erfolgen soll. Weiterhin sollen eine gute Planung und Aufteilung der anfallenden Arbeit auf die einzelnen Teammitglieder uns dabei helfen, unseren Produkt so zu erstellen, dass dieses schnell und effizient arbeitet.
