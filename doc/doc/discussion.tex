\section{Schlussdiskussion}

\subsection{Erfahrungen}

\subsubsection*{Daniel Mathis}
Zu Beginn war es wichtig, die Teammitglieder kennenzulernen, um so ein 
möglichst breites Vorwissen aufzubauen. Anhand dieses Vorwissens, 
konzentrierten wir uns schnell auf eine Lösung welche fliegen sollte. Die uns 
anschliessend in Auftrag gegebene Risikoanalyse zeigte jedoch auf, dass diese 
Lösung ein zu hohes Risiko darstellt. Dieses zeigte mir persönlich die 
Wichtigkeit eines strukturierten Ablaufes auf. Des weiteren war es nicht ganz 
einfach, die Aufgabenstellung zu unterteilen in die verschiedenen Teilsysteme 
und zu diesen Teilsystemen möglichst viele Möglichkeiten aufzuzeigen. 
Rückblickend betrachtend, gelang dies uns jedoch relativ gut. Eine weitere 
Schwierigkeit für mich waren die Berichte, welche abgegeben werden mussten. 
Deren Inhalt aus meiner Sicht nicht immer klar definiert war und uns daher 
etwas aus dem Zeitplan brachte. Im Grossen und Ganzen konnte ich jedoch viel 
von der Interdisziplinarität dieses Projektes profitieren und habe deshalb 
auch einiges über die Schnittstellen zwischen Mechanik und der Elektronik 
erlernt. Ich bin gespannt auf das nächste Semester, wenn wir unsere Ideen zu 
einer kompletten Maschine verbinden können.

\subsubsection*{Peter Kuonen}
Bereits im ersten Semester habe ich gute Erfahrungen mit dem Kontext-Modul erleben dürfen. Deshalb freute ich mich auf das PREN1-Modul besonders. Eine weitere Möglichkeit ein Interdisziplinäres Projekt umzusetzen. Es ist für mich immer spannend neue Menschen kennen zu lernen und die verschiedenen Fähigkeiten zu kombinieren. Beim PREN1 hatten wir wieder das Glück, dass sich die Gruppe auf Anhieb verstanden hat. Dies war für mich einer der wichtigsten Aspekte. Wenn das Team funktioniert, funktioniert auch das Projekt.
Was das Projekt selbst angeht, war ich etwas enttäuscht. Die Anfangs-Euphorie hat sich relativ bald gelegt. Die Aufgabe bestand lediglich darin Bälle in einen Korb zu versenken. Nach der Risikoanalyse stellte sich heraus, dass fliegende Lösungen keinen Sinn machen (zu viele Risiken) und dann blieb als beste Lösung ein stillstehender, sich drehender Turm. Im Bereich Informatik ist  kaum etwas zu tun. Im Vergleich zu vorhergehenden Jahren, bei der die Informatik auch einen beträchtlichen Beitrag zur Lösung leisten durfte.
Bei der Dokumentation und den Zwischenabgaben hatten wir unsere liebe Mühe. Teilweise wussten wir nicht genau was verlangt war. Nach mehreren Gesprächen mit dem Experten hat es  schlussendlich geklappt. Hier wären mir von Beginn an etwas mehr Informationen lieber gewesen. Da wir noch nie ein Projekt von diesem Umfang bearbeitet haben, ist die Führung durch die Dokumentation verloren gegangen. 
Im Grossen und Ganzen hat das PREN1 jedoch Spass gemacht und das Team hat wunderbar zusammen gearbeitet. Ich bin gespannt auf das nächste Semester, wenn wir unser Konzept zusammen im PREN2 umsetzen können.


\subsection{Offene Punkte/Risikien/Ausblick}
