\section{Konzept}

\subsection{Gesamtkonzept}
\begin{itemize}
    \item Schwungrad
    \item Kamera mit Bilderkennung
    \item Ausrichtung Turm
\end{itemize}

\subsection{Konzept Ballwurf}
\begin{itemize}
    \item Schwungrad
    \item Direktantrieb
    \item Ballnachführung
\end{itemize}
Der Ball wird mit einem Schwungrad beschleunigt. 

\subsection{Konzept Korbfindung}
\begin{itemize}
    \item Kamera
    \item Bilderkennung
\end{itemize}
Um den Korb zu finden wird eine Kamera verwendet. Das Kamerabild wird mit der 
freien Bibliothek OpenCV verarbeitet. Die Position des Korbes wird 
anschliessend an den Roboter geschickt. 

\subsection{Konzept Ausrichtung}
\begin{itemize}
    \item Turm
    \item Schrittmotor
    \item Endschalter
\end{itemize}
Der Roboter erhält vom Computer die Lage des Korbes auf dem Spielfeld. 
Daraufhin wird der Turm in die korrekte Lage gedreht um den Ball werfen zu 
können. Der Turm wird dabei mit einem Schrittmotor angetrieben. 

\subsection{Konzept Kommunikation}
\begin{itemize}
    \item Bluetooth zu Roboter
    \item WLAN zu Kamera
\end{itemize}
Das Kamerabild wird via WLAN zum PC übertragen. Die Kommunikation zum Roboter 
erfolgt über Bluetooth. Dabei wird eine Serielle Verbindung über Bluetooth 
verwendet. 


\subsection{Konzept Energieversorgung}
\begin{itemize}
    \item Bleiakku
    \item 24 .. 36 V
    \item Regler für 5 V, 3.3 V
\end{itemize}
Für die Versorgung des Roboters dient ein Bleiakku. 
