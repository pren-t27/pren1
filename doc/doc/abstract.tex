\section{Abstract}
Nachfolgende Dokumentation soll eine Übersicht über die im ersten Semester geleistete Arbeit verschaffen. Anhand der Dokumentation soll aufgezeigt werden wie vorgegangen wurde um die Ziele zu erreichen. Anhand der Aufgabenstellung wurden die Projekt- und Produktanforderungen, sowie die gruppenspezifischen Ziele definiert. In einem ersten Schritt wurde eine Technologierecherche gemacht. Ziel dieser Recherche war es, bereits vorhandene Technologien zu finden. Es wurde vorwiegend im Internet auf Youtube, Wikipedia oder in Internetforen recherchiert. Anhand dieser Recherchen wurden 3 Lösungskonzepte entwickelt, ein Flugobjekt (Quadcopter), ein Bodenobjekt fahrend und ein Bodenobjekt stehend (Drehturm). Anhand von verschiedenen Tests, Berechnungen und Risikoanalysen haben wir uns für die Variante Bodenobjekt stehend (Drehturm) entschieden. Anhand der im ersten Semester geleisteten Arbeit sollte die Herstellung des Gerätes im zweiten Semester gelingen.\\
Eine weitere Möglichkeit fürs Abstract.. Im oberen sind etwas viele "{}anhand{}" drin..\\
Die Aufgabe im Modul PREN1 besteht darin ein Konzept für einen Roboter zu entwickeln. Der Roboter soll autonom 5 Bälle in einen Korb, auf einem definierten Spielfeld, befördern. Es sollten verschiedene Konzepte und Lösungswege erarbeitet werden um eine optimale Kombination herauszuarbeiten. Um eine Grundlage zu schaffen, wurden in einem Brainstorming Ideen die zu einer Lösung führen zusammengetragen. Geeignete Ansätze wurden herausgefiltert und in einem weiteren Schritt ausgearbeitet und analysiert. Anhand der Ergebnisse wurden drei Lösungskonzepte ausgewählt. Ein fliegender Roboter, einer fahrend und einer stehend. Anhand von weiteren Risikoanalysen, Berechnungen und Teilfunktionsmustern ist man zum Schluss gekommen das ein stehender Roboter die effektivste und schnellste Lösung ist um die verlangte Problemstellung zu lösen. Mit dem finalen Konzept des stehenden Roboters wurde ein für uns geeignete Strategie gewählt, welche im PREN2 die erfolgreiche Umsetzung garantieren wird.