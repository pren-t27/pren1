\addcontentsline{toc}{section}{\protect\numberline{}Management Summary}
\section*{Management Summary}
Die Aufgabe im Modul PREN1 besteht darin ein Konzept für einen Roboter zu 
entwickeln. Der Roboter soll autonom 5 Bälle in einen Korb, auf einem 
definierten Spielfeld, befördern. Es sollten verschiedene Konzepte und 
Lösungswege erarbeitet werden um eine optimale Kombination herauszuarbeiten. 
Um eine Grundlage zu schaffen, wurden in einem Brainstorming Ideen die zu 
einer Lösung führen zusammengetragen. Geeignete Ansätze wurden herausgefiltert 
und in einem weiteren Schritt ausgearbeitet und analysiert. Anhand der 
Ergebnisse wurden drei Lösungskonzepte ausgewählt. Ein fliegender Roboter, 
einer fahrend und einer stehend. Anhand von weiteren Risikoanalysen, 
Berechnungen und Teilfunktionsmustern ist man zum Schluss gekommen das ein 
stehender Roboter die effektivste und schnellste Lösung ist um die verlangte 
Problemstellung zu lösen. Mit dem finalen Konzept des stehenden Roboters wurde 
ein für uns geeignete Strategie gewählt, welche im PREN2 die erfolgreiche 
Umsetzung garantieren wird.
