\addcontentsline{toc}{section}{\protect\numberline{}Management Summary}
\section*{Management Summary}
Die Aufgabe im Modul PREN1 besteht darin ein Konzept für ein Gerät zu 
entwickeln. Das Gerät soll auf einem definierten Spielfeld autonom 
fünf Bälle in einen Korb befördern. Es sollen verschiedene Konzepte und 
Lösungswege erarbeitet werden, damit sich eine optimale Kombination herauskristallisiert. 
Um eine Grundlage zu schaffen, wurden in einem Brainstorming Ideen, die zu 
einer Lösung führen, zusammengetragen. Geeignete Ansätze wurden herausgefiltert 
und in einem weiteren Schritt ausgearbeitet und analysiert. Anhand der 
Ergebnisse wurden drei Lösungskonzepte ausgewählt: Ein fliegendes, 
ein fahrendes und ein stehendes Gerät. Anhand von weiteren Risikoanalysen, 
Berechnungen und Teilfunktionsmustern ist man zum Schluss gekommen, dass ein 
stehendes Gerät die effektivste und schnellste Lösung ist, um die verlangte 
Problemstellung zu lösen. Mit dem finalen Konzept des stehenden Gerät wurde 
eine für uns geeignete Strategie gewählt, welche im PREN2 die erfolgreiche 
Umsetzung garantiert.
