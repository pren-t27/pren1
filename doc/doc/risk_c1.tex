\subsubsection{Risikoanalyse Konzept 1 (Bodenobjekt fahrend)}
\begin{table}[h!]
    \centering
    \begin{zebratabular}{@{}cp{0.25\linewidth}llp{0.25\linewidth}}      
        \textbf{ID}&\textbf{Beschreibung}&\textbf{Wahrscheinlichkeit}&\textbf{Auswirkung}&\textbf{Massnahmen}\\
        \hline
        \#17&Fahrmechanismus funktioniert nicht&niedrig&hoch&verschiedene Fahrmechanismen testen\\
        \#18&Ausrichten funktioniert nicht&niedrig&sehr hoch&Testobjekt bauen, möglichst einfach Konstruktion\\
        \#19&Balllager (Nachschieben) streikt&sehr niedrig&hoch&Mechanismus einbauen, der nicht blockiert\\
        \#20&Umfallen durch Rückstoss&sehr niedrig&niedrig&breiterer Stand, Rückstoss berechnen\\       
    \end{zebratabular}
    \caption{Risikoanalyse Bodenobjekt fahrend}
\end{table}
