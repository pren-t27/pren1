\subsection{Risikoanalye allgemein}
\begin{table}[h!]
	\begin{zebratabular}{@{}clllp{0.25\linewidth}}		
		\textbf{ID}&\textbf{Beschreibung}&\textbf{Wahrscheinlichkeit}&\textbf{Auswirkung}&\textbf{Massnahmen}\\
		\hline
		\#1&Teammitglied Elektrotechnik fällt aus&sehr niedrig&sehr hoch&keine\\
		\#2&Teammitglied Maschinenbau fällt aus&sehr niedrig&hoch&Übernahme durch Maschinenbauer\\
		\#3&Teammitglied Informatik fällt aus&sehr niedrig&hoch&Übernahme durch Informatiker\\
		\#4&Budget wird überschritten&niedrig&hoch&Vorab Kosten abklären\\
		\#5&Zeit reicht nicht zur Realisierung&mittel&sehr hoch&Zeitplanung mit Meilensteinen\\
		\#6&Eingekaufte Komponente fällt aus&niedrig&hoch&Neu bestellen\\
		\#7&Komponente (Eigenbau) fällt aus&niedrig&mittel&Neu bauen\\
		\#8&Komponente funktioniert nicht&niedrig&mittel&Komponente reparieren\\
		\#9&Berechnungen falsch&sehr niedrig&hoch&Überprüfung durch mehrere Personen\\
		\#10&Abmessungseinschränkungen überschritten&sehr niedrig&hoch&Modell bauen\\
		\#11&Schnittstellen ungenau definiert&sehr niedrig&sehr hoch&Überprüfung durch mehrere Personen\\
		\#12&Startsignal funktioniert nicht&niedrig&hoch&Ausgiebiges Testen\\
		\#13&Endsignal nicht übermittelt&niedrig&sehr niedrig&Ausgiebiges Testen\\
		\#14&Verfehlen des Korbs&mittel&mittel&Ausgiebiges Testen\\
		\#15&Stromversorgung reicht nicht aus&sehr niedrig&mittel&Ausgiebiges Testen\\
		\#16&Anforderungen ändern&sehr niedrig&hoch&Neu planen\\
	\end{zebratabular}
\end{table}
