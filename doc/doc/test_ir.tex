\subsection{Infrarot Sensor GP2Y0A710K0F}
Diese Messung wird gemeinsam mit Gruppe 39 durchgeführt. \\
Das Modul GP2Y0A710K0F von Sharp ist ein Infrarotsensor, mit welchem Distanzen 
gemessen werden können. Die Messung basiert auf Triangulation. Der Sensor 
beinhaltet eine Infrarot LED (Light emitting diode) mit einer Linse. Das vom 
Objekt reflektierte Licht fällt über eine weitere Linse auf eine Reihe von 
Photoempfängern. Damit wird gemessen, unter welchem Winkel das reflektierte 
Licht auf den Sensor trifft. Die Distanz lässt sich wie folgt berechnen: 
\[ D = d \cdot \arctan(\alpha) \]
\begin{tabular}{@{}ll}
    $D$         & Distanz zum Objekt \\
    $d$         & Abstand der Linsen (38mm) \\
    $\alpha$    & Winkel des zurückreflektierten Lichts gegenüber der Sensorebene \\
\end{tabular} \\
Preislich bewegt sich der Sensor im selben Rahmen wie Ultraschall Module. 

\subsubsection{Eckdaten}
\begin{table}[h!]
    \centering
    \begin{zebratabular}{ll}
        \rowcolor{gray} Eigenschaft & Wert \\
        Messbereich                 & 1 - 5.5m \\
        Interface                   & Analog \\
        Strombedarf                 & <50mA \\
        Spannung                    & 4.5 - 5.5V \\
    \end{zebratabular}
    \caption[Eckdaten GP2Y0A710K0F]{Eckdaten}
\end{table}

\subsubsection{Testkorb}
Als Testkorb wird der Abfallkorb aus dem Raum C200 verwendet. \\
\begin{table}[h!]
    \centering
    \begin{zebratabular}{ll}
        \rowcolor{gray} Eigenschaft & Wert \\
        Durchmesser oben    & 38 cm \\
        Durchmesser unten   & 33 cm \\
        Höhe                & 48 cm \\
        Farbe               & schwarz (matt) \\
        Material            & Kunststoff () \\
        Hersteller          & Helit \\
        Typ                 & 61062 \\
    \end{zebratabular}
    \caption{Definition Testkorb}
\end{table}

\subsubsection{Messmittel}
\begin{table}[h!]
    \centering
    \begin{zebratabular}{lll}
        \rowcolor{gray} Gerät &
            Typ &
            Nummer \\
        Speisegerät &
            Hameg ... &
            SN ... \\
        Mainframe &
            Hameg 800? &
            SN ... \\
        Oszilloskop &
            Agilent &
            SN ... \\
        Signalgenerator Servo &
            SM Modellbau Unitest 2 &
            SN: 25494.3 \\
    \end{zebratabular} \\
    \caption[Messmittel Messungen GP2Y0A710K0F]{Messmittel}
\end{table}
Die Messungen werden im Raum B332c durchgeführt. Die Messstrecke liegt dabei 
auf der ersten Tischreihe. Der Sensor ist am Ende der Wand montiert. 

\subsubsection{Messung Abtastung}
Um die Funktionsweise des Sensors besser zu verstehen, wird zunächst die 
Abtastung des Sensors ermittelt. Dazu wird die ausgesandte Infrarotstrahlung 
mittels einer Photodiode (SFH213) gemessen. Um die Diode zu entladen, wird ein 
Widerstand mit 100 k$\Omega$ parallel an die Diode angeschlossen. 

Die Messung zeigt, dass der Sensor mit Impulspaketen arbeitet. Die Pulse haben 
eine Pulsweite von 155 $\mu$s und eine Periodendauer von 1 ms. Ein Impulspaket 
besteht aus acht Pulsen und wird alle 16 ms wiederholt. Dieser Wert deckt sich 
mit dem vom Hersteller engegebenen Wert von 16.5 ms $\pm$ 3.7 ms. vgl. 
\cite{Datasheet:GP2Y0A710K0F}

\subsubsection{Messung Messgenauigkeit}
Die Messung der Messgenauigkeit wird bei geschlossenen Storen durchgeführt. Zudem 
ist die Beleuchtung bis auf die Tafelbeleuchtung eingeschaltet. Die Werte 
werden aus mindestens 100 Messungen bestimmt. \\
\begin{table}[h!]
    \centering
    \begin{zebratabular}{lll}
        \rowcolor{gray} Abstand [cm] & Vout (mean) [V] & Std. Dev [mV] \\
        50  & 3.09  & 542 \\
        60  & 3.09  & 1.15 \\
        70  & 2.94  & 6.8 \\
        80  & 2.71  & 10.7 \\
        90  & 2.46  & 19.0 \\
        100 & 2.18  & 27.0 \\
        110 & 2.33  & 11.6 \\
        120 & 1.83  & 26.2 \\
        130 & 1.58  & 23.6 \\
        140 & 1.58  & 29.4 \\
        150 & 1.33  & 30.0 \\
        160 & 1.21  & 33.5 \\
        170 & 0.979 & 44.5 \\
        180 & 0.871 & 32.1 \\
        190 & 0.818 & 24.7 \\
        200 & 0.776 & 26.1 \\
        210 & 0.809 & 24.1 \\
        220 & 0.868 & 29.7 \\
    \end{zebratabular} \\
    \caption[Messwerte Messgenauigkeit GP2Y0A710K0F]{Messwerte Messgenauigkeit}
\end{table}
Bei den Messungen zeigt sich, dass das Abtastinterval im Signal sichtbar ist 
und somit nicht gefiltert wird. \\
Ausserdem zeigt sich, dass das Ergebnis stark vom seitlichen 
Versatz des Eimers gegenüber der Sensorachse abhängig ist. 

\subsubsection{Messung Durchfahren eines Objektes}
Der Eimer wird vor dem Sensor durchgefahren. Anschliessend wird der Eimer mit 
weissem Papier beklebt und der Versuch erneut durchgeführt. Der Eimer hat 
dabei eine Distanz von 2 m zum Sensor.  Die Wand hinter dem Sensor hat einen 
Abstand von 2.6 m zum Sensor. 
\begin{figure}[h!]
    \begin{minipage}{0.5\textwidth}
        \centering
        \includegraphics[width=0.8\textwidth]{fig/scope_75.png}
        \caption*{Schwarzer Korb}
        \label{fig:shift_ir_black}
    \end{minipage}
    \begin{minipage}{0.5\textwidth}
        \centering
        \includegraphics[width=0.8\textwidth]{fig/scope_77.png}
        \caption*{Weisser Korb}
        \label{fig:shift_ir_white}
    \end{minipage}
    \caption{Durchfahren des Korbes}
    \label{fig:shift_ir}
\end{figure}

\subsubsection{Messung Scan mit Servomotor}
Als letzter Versuch wird der Sensor auf ein Servo montiert und hin und her 
gedreht. Damit wird ein Scan, wie er in einer Anwendung durchgeführt wird, 
nachgebildet. Die Dauer eines Scans beträgt 0.72 s. Der Abstand zwischen 
Eimer und Sensor beträgt 1.39 m. Die Wand hinter dem Eimer hat eine Distanz 
von 2 m zum Sensor. Auch dieser Versuch wird mit einem schwarzen und mit einem 
weissen Eimer durchgeführt. 
\begin{figure}[h!]
    \begin{minipage}{0.5\textwidth}
        \centering
        \includegraphics[width=0.8\textwidth]{fig/scope_80.png}
        \caption{Schwarzer Korb}
        \label{fig:scan_ir_black}
    \end{minipage}
    \begin{minipage}{0.5\textwidth}
        \centering
        \includegraphics[width=0.8\textwidth]{fig/scope_82.png}
        \caption{Weisser Korb}
        \label{fig:scan_ir_white}
    \end{minipage}
    \caption{Scan mit Korb}
    \label{fig:scan_ir}
\end{figure}

\subsubsection{Fazit}
Der Sensor ist nicht geeignet, um den Korb zu finden. Aufgrund der schwarzen 
Farbe und der runden Form wird der Korb nur in der Mitte erkannt. An den 
Rändern ist die Reflektion in Richtung Sensor zu schwach um den Korb erkennen 
zu können. 

