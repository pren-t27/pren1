\begin{landscape}
\pagestyle{empty}

\newcommand{\morphcellwidth}
{
0.12\linewidth
}
\section{Lösungskonzepte}
Mit diesem Abschnitt sollen die erarbeiteten Konzepte gegenübergestellt und 
anhand von vorgegebenen Kriterien bewertet werden. Dies ergibt eine 
Auswahl der optimalen Kombination, welche in einem nächsten Schritt 
ausgearbeitet wird. Das daraus resultierende, definitive Konzept wird im nachfolgenden Kapitel "Produktbeschreibung Funktion" näher erläutert. 
Das im Anhang beigefügte Dokument "'Brainstorming"' zeigt die ersten Ideen für 
die verschiedenen Funktionsweisen. Hierbei wurde weder auf Risiken, Kosten 
noch auf Umsetzbarkeit geachtet. In einem weiteren Schritt werden die 
sinnvollsten Kombinationen in morphologischen Kästen kombiniert und für die 
jeweiligen Lösungen erweitert und ausgewertet. 


\subsection{Variante Bodenobjekt fahrend}
\subsubsection{Kurzbeschrieb}
Bei diesem Konzept soll die Abschussvorrichtung fahrend sein. Hierbei kann 
die  Wurfdistanz zum Ziel verringert werden und auch das Zielen mittels 
Bewegung erfolgen.

\subsubsection{Morphologischer Kasten}
% Morphologischer Kasten mit Stichworten
\footnotesize
\begin{table}[h!]
    %\begin{zebratabular}{@{}p{0.2\linewidth}p{\morphcellwidth}p{\morphcellwidth}p{\morphcellwidth}p{\morphcellwidth}p{\morphcellwidth}p{\morphcellwidth}}
    \begin{zebratabular}{@{}p{0.2\linewidth}llllll}
        \rowcolor{gray}
        Eigenschaften &
            \multicolumn{6}{c}{Merkmalausprägung} \\
        Art der Bewegung &
            Fliegend                \tikzmark{1move:fly}          &
            Fahrend                 \tikzmark{1move:drive}        &
            Stehend                 \tikzmark{1move:stay}         &
                                    \tikzmark{1move:none3}        &
                                    \tikzmark{1move:none2}        &
                                    \tikzmark{1move:none1}        \\
        Ballbeförderung &
            Ballistisch             \tikzmark{1ballmove:ball}     &
            Abwurf                  \tikzmark{1ballmove:drop}     &
                                    \tikzmark{1ballmove:none4}    &
                                    \tikzmark{1ballmove:none3}    &
                                    \tikzmark{1ballmove:none2}    &
                                    \tikzmark{1ballmove:none1}    \\
        Erkennen des Korbes &
            Distanzmessung          \tikzmark{1localize:dist}     &
            Optische Erkennung      \tikzmark{1localize:opt}      &
                                    \tikzmark{1localize:none4}    &
                                    \tikzmark{1localize:none3}    &
                                    \tikzmark{1localize:none2}    &
                                    \tikzmark{1localize:none1}    \\
        Ballager &
            Einzeln                 \tikzmark{1store:single}      &
            Zusammen                \tikzmark{1store:mult}        &
                                    \tikzmark{1store:none4}       &
                                    \tikzmark{1store:none4}       &
                                    \tikzmark{1store:none4}       &
                                    \tikzmark{1store:none1}       \\
        Übertragung Startsignal &
            Funk                    \tikzmark{1connect:radio}     &
            Optisch                 \tikzmark{1connect:opt}       &
            Ultraschall             \tikzmark{1connect:us}        &
            Spracherkennung         \tikzmark{1connect:voice}     &
                                    \tikzmark{1connect:none2}     &
                                    \tikzmark{1connect:none1}     \\
        Energieversorgung &
            intern elektrisch       \tikzmark{1power:elint}       &
            extern elektrisch       \tikzmark{1power:elext}       &
            Verbrennungsmotor       \tikzmark{1power:combustion}  &
            Druckluft               \tikzmark{1power:air}         &
            Potenzielle Energie     \tikzmark{1power:height}      &
            Feder                   \tikzmark{1power:spring}      \\
    \end{zebratabular}
\end{table}

% Linien zur Darstellung von Lösungsvarianten
\begin{tikzpicture}[remember picture,overlay]
    \draw [ultra thick, rounded corners, red] 
        ({pic cs:1move:drive}) circle(2pt)
        --
        ({pic cs:1ballmove:ball}) circle(2pt)
        --
        ({pic cs:1localize:dist}) circle(2pt)
        --
        ({pic cs:1store:single}) circle(2pt)
        --
        ({pic cs:1connect:radio}) circle(2pt)
        --
        ({pic cs:1power:elint}) circle(2pt)
        ;
\end{tikzpicture}
\normalsize

\subsubsection{Vor- und Nachteile}
\begin{minipage}{\textwidth}
    \begin{itemize}
    	\item[+] auffallend
    	\item[+] Distanz zum Korb ist immer gleich
    	\item[-] langsam (lange Ausrichtzeit)
    	\item[-] aufwändige Konstruktion
    	\item[-] hohes Gewicht
    \end{itemize}
\end{minipage}

\subsubsection{Funktionsweise}
Eine fahrende Lösung wäre durch das anfängliche Ausrichten des Geräts und 
den anschliessenden Wurf ein technisch anspruchsvoller und auch interessanter 
Ansatz. Jedoch wird gerade aufgrund dieser Punkte das erreichen der 
Projektziele erschwert. Das Fahren benötigt im Vergleich zu einem direkten 
Wurf, welcher durch die kurze Distanz gut realisierbar wäre, viel Zeit. Auch 
wird die Konstruktion durch die Fortbewegung komplizierter und somit auch 
teurer und schwerer. Durch die klare Aufgabenstellung ist auch eine 
universellere Nutzung mit eventueller Erweiterungsmöglichkeiten nicht nötig. 

\subsubsection{Risikoanalyse Konzept 1 (Bodenobjekt fahrend)}
\begin{table}[h!]
	\begin{zebratabular}{@{}cp{0.25\linewidth}llp{0.25\linewidth}}		
		\textbf{ID}&\textbf{Beschreibung}&\textbf{Wahrscheinlichkeit}&\textbf{Auswirkung}&\textbf{Massnahmen}\\
		\hline
		\#17&Fahrmechanismus funktioniert nicht&niedrig&hoch&Ausgiebiges Testen\\
		\#18&Ausrichten funktioniert nicht&niedrig&sehr hoch&Ausgiebiges Testen\\
		\#19&Balllager (Nachschieben) streikt&sehr niedrig&hoch&Mechanismus einbauen, der nicht blockiert\\
		\#20&Umfallen durch Rückstoss&sehr niedrig&niedrig&Ausgiebiges Testen\\		
	\end{zebratabular}
\end{table}


\clearpage

\subsection{Variante Flugobjekt}
\subsubsection{Kurzbeschrieb}
Mit der Wahl eines fliegenden Objektes kann viel Aufsehen erregt werden. 
Allerdings gestaltet sich die Konstruktion auch schwieriger als bei anderen 
Geräten. Ebenfalls ein wichtiger Faktor ist die Herausforderung für die Gruppe 
erfolgreich ein fliegendes Objekt zu bauen. Da im Handel schon 
unterschiedliche Flugobjekte preiswert erworben werden können, scheint die 
Aufgabe einfacher, als das sie es wirklich ist.

\subsubsection{Morphologischer Kasten}
% Morphologischer Kasten mit Stichworten
\footnotesize
\begin{table}[h!]
    %\begin{zebratabular}{@{}p{0.2\linewidth}p{\morphcellwidth}p{\morphcellwidth}p{\morphcellwidth}p{\morphcellwidth}p{\morphcellwidth}p{\morphcellwidth}}
    \begin{zebratabular}{@{}p{0.2\linewidth}llllll}
        \rowcolor{gray}
        Eigenschaften &
            \multicolumn{6}{c}{Merkmalausprägung} \\
        Art der Bewegung &
            Fliegend                \tikzmark{2move:fly}          &
            Fahrend                 \tikzmark{2move:drive}        &
            Stehend                 \tikzmark{2move:stay}         &
                                    \tikzmark{2move:none3}        &
                                    \tikzmark{2move:none2}        &
                                    \tikzmark{2move:none1}        \\
        Ballbeförderung &
            Ballistisch             \tikzmark{2ballmove:ball}     &
            Abwurf                  \tikzmark{2ballmove:drop}     &
                                    \tikzmark{2ballmove:none4}    &
                                    \tikzmark{2ballmove:none3}    &
                                    \tikzmark{2ballmove:none2}    &
                                    \tikzmark{2ballmove:none1}    \\
        Erkennen des Korbes &
            Distanzmessung          \tikzmark{2localize:dist}     &
            Optische Erkennung      \tikzmark{2localize:opt}      &
                                    \tikzmark{2localize:none4}    &
                                    \tikzmark{2localize:none3}    &
                                    \tikzmark{2localize:none2}    &
                                    \tikzmark{2localize:none1}    \\
        Ballager &
            Einzeln                 \tikzmark{2store:single}      &
            Zusammen                \tikzmark{2store:mult}        &
                                    \tikzmark{2store:none4}       &
                                    \tikzmark{2store:none4}       &
                                    \tikzmark{2store:none4}       &
                                    \tikzmark{2store:none1}       \\
        Übertragung Startsignal &
            Funk                    \tikzmark{2connect:radio}     &
            Optisch                 \tikzmark{2connect:opt}       &
            Ultraschall             \tikzmark{2connect:us}        &
            Spracherkennung         \tikzmark{2connect:voice}     &
                                    \tikzmark{2connect:none2}     &
                                    \tikzmark{2connect:none1}     \\
        Energieversorgung &
            intern elektrisch       \tikzmark{2power:elint}       &
            extern elektrisch       \tikzmark{2power:elext}       &
            Verbrennungsmotor       \tikzmark{2power:combustion}  &
            Druckluft               \tikzmark{2power:air}         &
            Potenzielle Energie     \tikzmark{2power:height}      &
            Feder                   \tikzmark{2power:spring}      \\
    \end{zebratabular}
\end{table}

% Linien zur Darstellung von Lösungsvarianten
\begin{tikzpicture}[remember picture,overlay]
    \draw [ultra thick, rounded corners, red] 
        ({pic cs:2move:fly}) circle(2pt)
        --
        ({pic cs:2ballmove:drop}) circle(2pt)
        --
        ({pic cs:2localize:opt}) circle(2pt)
        --
        ({pic cs:2store:mult}) circle(2pt)
        --
        ({pic cs:2connect:radio}) circle(2pt)
        --
        ({pic cs:2power:elint}) circle(2pt)
        ;
\end{tikzpicture}
\normalsize

\subsubsection{Vor- und Nachteile}
\begin{minipage}{\textwidth}
    \begin{itemize}
    	\item[+] sehr leicht
    	\item[+] auffallend
    	\item[+] schnell, da alle Bälle gleichzeitig transportiert werden
    	\item[-] sehr aufwändige programmierung
    	\item[-] hohes Risiko (minimaler Fehler führt zum scheitern der Aufgabe)
    \end{itemize}
\end{minipage}

\subsubsection{Funktionsweise}
Mit einem fliegenden Objekt kann die Aufgabe vom Prinzip her relativ einfach bewältigt 
werden. Allerdings wird es schwierig eine gut funktionierende Regelung zu entwickeln, um da Flugobjekt ruhig in der Luft zu halten und die gewünschte Position genau anzufliegen. 
Vom Startpunkt aus bewegt sich das Objekt in der Luft und erkennt den Korb optisch 
mit einer Kamera und fliegt anschliessend über diesen und wirft dort die Bälle 
ab. Die Bälle müssten nicht einzeln abgeworfen werden, sie könnten alle zusammen 
in einem entsprechenden Behälter transportiert und abgeworfen 
werden. Sollte der Korb korrekt erkannt worden sein und das Flugobjekt sich 
richtig über ihn bewegt haben, kann eine hundertprozentige Erfolgsquote 
erreicht werden. Sollte jedoch nur eine dieser Bedingungen nicht erfüllt werden, verfehlen  
automatisch sämtliche Bälle das Ziel und die Aufgabe wird nicht erfüllt. 
Bezüglich des Gewichts hat ein Flugobjekt grosse Vorteile. Da das Eigengewicht 
sowieso tief gehalten werden muss, wird es einfach sein, das Gesamtgewicht 
unter zwei Kilogramm zu halten.
%
Aufgrund der Aufgabenstellung muss sich das Flugobjekt autonom bewegen, es 
wird viel Aufwand benötigt um dieses kontrolliert fliegen zu lassen. Es muss 
sich selbstständig positionieren und auch vor dem möglichen wegdriften bewahrt 
werden. Zudem muss beachtet werden, dass ein Flugobjekt eine gewisse Zeit 
braucht, um den Korb zu erkennen und sich anschliessend auch passend darüber 
zu positionieren. Am Meisten gegen ein Flugobjekt spricht jedoch die 
Risikoanalyse, es gibt sehr viele Faktoren, welche kontrolliert und überwacht 
werden müssen. Dies ist sowohl zeit- als auch rechenintensiv. 

\subsection{Risikoanalye Konzept 2 (Bodenobjekt stehend)}
\begin{table}[h!]
	\begin{zebratabular}{@{}clllp{0.25\linewidth}}		
		\textbf{ID}&\textbf{Beschreibung}&\textbf{Wahrscheinlichkeit}&\textbf{Auswirkung}&\textbf{Massnahmen}\\
		\hline
		\#22&Drehung funktioniert nicht&sehr niedrig&mittel&Ausgiebiges Testen\\
		\#23&Abschussmechanismus fällt aus&sehr niedrig&hoch&Ausgiebiges Testen\\
		\#24&Balllager (Nachschieben) streikt&sehr niedrig&hoch&Mechanismus einbauen, der nicht blockiert\\
		\#25&Umfallen durch Rückstoss&sehr niedrig&niedrig&Ausgiebiges Testen\\
		\#26&Genauigkeit des Schusses&mittel&hoch&Gleichmässige Beschleunigung garantieren\\
	\end{zebratabular}
\end{table}


\clearpage

\subsection{Variante Bodenobjekt stehend}
\subsubsection{Kurzbeschrieb}
Hierbei handelt es sich um eine Variante die nicht fahren kann. Die Schussvorrichtung ist rotativ beweglich.

\subsubsection{Morphologischer Kasten}
% Morphologischer Kasten mit Stichworten
\footnotesize
\begin{table}[h!]
    %\begin{zebratabular}{@{}p{0.2\linewidth}p{\morphcellwidth}p{\morphcellwidth}p{\morphcellwidth}p{\morphcellwidth}p{\morphcellwidth}p{\morphcellwidth}}
    \begin{zebratabular}{@{}p{0.2\linewidth}llllll}
        \rowcolor{gray}
        Eigenschaften &
            \multicolumn{6}{c}{Merkmalausprägung} \\
        Art der Bewegung &
            Fliegend                \tikzmark{3move:fly}          &
            Fahrend                 \tikzmark{3move:drive}        &
            Stehend                 \tikzmark{3move:stay}         &
                                    \tikzmark{3move:none3}        &
                                    \tikzmark{3move:none2}        &
                                    \tikzmark{3move:none1}        \\
        Ballbeförderung &
            Ballistisch             \tikzmark{3ballmove:ball}     &
            Abwurf                  \tikzmark{3ballmove:drop}     &
                                    \tikzmark{3ballmove:none4}    &
                                    \tikzmark{3ballmove:none3}    &
                                    \tikzmark{3ballmove:none2}    &
                                    \tikzmark{3ballmove:none1}    \\
        Erkennen des Korbes &
            Distanzmessung          \tikzmark{3localize:dist}     &
            Optische Erkennung      \tikzmark{3localize:opt}      &
                                    \tikzmark{3localize:none4}    &
                                    \tikzmark{3localize:none3}    &
                                    \tikzmark{3localize:none2}    &
                                    \tikzmark{3localize:none1}    \\
        Ballager &
            Einzeln                 \tikzmark{3store:single}      &
            Zusammen                \tikzmark{3store:mult}        &
                                    \tikzmark{3store:none4}       &
                                    \tikzmark{3store:none4}       &
                                    \tikzmark{3store:none4}       &
                                    \tikzmark{3store:none1}       \\
        Übertragung Startsignal &
            Funk                    \tikzmark{3connect:radio}     &
            Optisch                 \tikzmark{3connect:opt}       &
            Ultraschall             \tikzmark{3connect:us}        &
            Spracherkennung         \tikzmark{3connect:voice}     &
                                    \tikzmark{3connect:none2}     &
                                    \tikzmark{3connect:none1}     \\
        Energieversorgung &
            intern elektrisch       \tikzmark{3power:elint}       &
            extern elektrisch       \tikzmark{3power:elext}       &
            Verbrennungsmotor       \tikzmark{3power:combustion}  &
            Druckluft               \tikzmark{3power:air}         &
            Potenzielle Energie     \tikzmark{3power:height}      &
            Feder                   \tikzmark{3power:spring}      \\
    \end{zebratabular}
\end{table}

% Linien zur Darstellung von Lösungsvarianten
\begin{tikzpicture}[remember picture,overlay]
    \draw [ultra thick, rounded corners, red] 
        ({pic cs:3move:stay}) circle(2pt)
        --
        ({pic cs:3ballmove:ball}) circle(2pt)
        --
        ({pic cs:3localize:opt}) circle(2pt)
        --
        ({pic cs:3store:single}) circle(2pt)
        --
        ({pic cs:3connect:radio}) circle(2pt)
        --
        ({pic cs:3power:elint}) circle(2pt)
        ;
\end{tikzpicture}
\normalsize

\subsubsection{Vor- und Nachteile}
\begin{minipage}{\textwidth}
    \begin{itemize}
        \item[+] einfachere Mechanik
        \item[+] durch einfachere Mechanik geringeres Gewicht
        \item[+] sehr schnell
        \item[+] bessere Bodenhaftung für Rückschlag 
        \item[-] Schussgenauigkeit muss gewährleistet sein
        \item[-] wenig Spielraum mit der Balldicke und dem Balldruck
        \item[-] Bälle werden nacheinander geschossen und ist somit zeitaufwändiger
    \end{itemize}
\end{minipage}

\subsubsection{Funktionsweise}
Dieses Gerät lokalisiert den Korb mit optischen Sensoren und richtet die 
Schussvorrichtung danach aus. Der Abschusswinkel bleibt fix, die Richtung und 
Schussweite können eingestellt werden. Die Bälle werden mit einem 
ballistischen Mechanismus abgeschossen. Angenommen das Gerät wird im Startfeld 
mittig an der Startlinie positioniert, und der Korb wird ganz am Rand des 
Feldes positioniert, ergibt sich für die maximale Wurflänge 1992.49mm. Die 
minimale Wurflänge (Korb wird mittig positioniert) beträgt 1900mm. Da der Korb 
einen Durchmesser von 30cm hat, könnte somit immer gleich weit geschossen 
werden, vorausgesetzt die Schussweitentoleranz beträgt weniger als ±10cm 
(siehe Dokument "'Berechnungen"' im Anhang). Dies ist deshalb auch die grösste 
Schwierigkeit bei dieser Variante. Da nur eine drehbare Schussvorrichtung nötig 
ist, wird die Mechanik stark vereinfacht, dadurch kann der Fokus auf den 
Leichtbau gelegt werden, so dass das Projektziel das Gewicht unter 2kg zu 
halten, erreicht werden sollte. Die Bälle werden nacheinander geschossen, wie 
schnell dies passieren kann, ist abhängig von der Konstruktion und von der 
Massenträgheit des Antriebes und dem Drehmoment des Motors. Nach ersten 
Experimenten kann jedoch davon ausgegangen werden, dass die Frequenz sehr hoch 
ist.

\clearpage

\subsection{Risikoanalye Konzept 3 (Bodenobjekt stehend)}
\begin{table}[h!]
	\begin{zebratabular}{@{}clllp{0.25\linewidth}}		
		\textbf{ID}&\textbf{Beschreibung}&\textbf{Wahrscheinlichkeit}&\textbf{Auswirkung}&\textbf{Massnahmen}\\
		\hline
		\#22&Drehung funktioniert nicht&sehr niedrig&mittel&Ausgiebiges Testen\\
		\#23&Abschussmechanismus fällt aus&sehr niedrig&hoch&Ausgiebiges Testen\\
		\#24&Balllager (Nachschieben) streikt&sehr niedrig&hoch&Mechanismus einbauen, der nicht blockiert\\
		\#25&Umfallen durch Rückstoss&sehr niedrig&niedrig&Ausgiebiges Testen\\
		\#26&Genauigkeit des Schusses&mittel&hoch&Gleichmässige Beschleunigung garantieren\\
	\end{zebratabular}
\end{table}


\clearpage

\end{landscape}

\subsection{Gegenüberstellung und Bewertung/Fazit}
\begin{tabular}{@{}p{0.15\linewidth}p{0.75\linewidth}}
    Gewicht:        & Wahrscheinlichkeit unter 2kg zu bleiben. \\
    Dauer:          & Zeit, bis Aufgabe erledigt ist \\
    Genauigkeit:    & Wahrscheinlichkeit, dass alle Bälle versenkt werden \\
    Risikofaktor:   & hoher Faktor entspricht einer weniger hohen Ausfallwahrscheinlichkeit. \\
    Aufwand:        & Arbeit die investiert werden muss, um das Projekt umzusetzen (höherer Faktor entspricht weniger Aufwand) \\
\end{tabular}

\begin{table}[h!]
    \begin{zebratabular}{lccc}
        \rowcolor{gray}         & Bodenobjekt fahrend   & Flugobjekt    & Bodenobjekt stehend   \\
        Gewicht                 & 2                     & 10            & 8                     \\
        Dauer                   & 3                     & 6             & 7                     \\
        Genauigkeit             & 6                     & 5             & 7                     \\
        Risikofaktor            & 6                     & 3             & 7                     \\
        Aufwand                 & 5                     & 4             & 9                     \\
        \rowcolor{gray} Total   & 22                    & 29            & 38                    \\
    \end{zebratabular}
\end{table}

Nach der detaillierten Ausarbeitung der einzelnen Lösungsprinzipien sind wir 
zum Schluss gekommen, dass das stationär Gerät mit drehbarer 
Abschussvorrichtung (im folgenden „Drehturm“ genannt) die ideale Lösung für 
unsere Gruppe ist und so die Projektziele bezüglich Gewicht und 
Geschwindigkeit am Besten erreicht werden können. Durch die einfache Bauweise 
kann Gewicht eingespart werden und durch die simple Funktionsweise wird das 
Risiko verringert. Für den Drehturm spricht ebenfalls, dass er in keinem der 
von uns bewerteten Punkte eine schlechte Bewertung aufweist, dies ist bei den 
anderen Lösungsprinzipien der Fall. Das Flugobjekt schneidet besonders in der 
Risikobeurteilung schlecht ab und das fahrende Objekt sammelt beim Gewicht 
viele Negativpunkte. Durch einfache Berechnungen konnten ein Teil der 
benötigten Leistungsparameter abgeschätzt werden. Der Drehturm präsentiert 
sich somit als solideste Gesamtlösung.

