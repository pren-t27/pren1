\section{Benötigte Funktionen}
Aus der Aufgabenstellung können nun die für das Lösen der Aufgabe benötigten 
Funktionen extrahiert werden. Vor dem Start müssen die Bälle in einem Balllager 
aufgenommen werden. Nach dem Start muss als Erstes das Startsignal drahtlos 
übermittelt werden. Danach muss der Korb gefunden und angepeilt werden. 
Anschliessend müssen die Bälle in den Korb befördert werden. Ausserdem muss 
das Gerät mit Energie versorgt werden. Also ergeben sich folgende benötigte 
Funktionen: 

Aus der Aufgabenstellung können die für das Lösen der Aufgabe benötigten Funktionen extrahiert werden.
Diese bestehen aus einem Balllager welches die Tennisbälle vor dem Start aufnimmt. Des weiteren wird,
um das Gerät arbeiten zu lassen, ein Startsignal benötigt welches Drahtlos übertragen werden soll. Weitere
Teilfunktionen sind die Art der Fortbewegung und die Korberkennung. Je nach Art der Fortbewegung wird eine 
andere Erkennung benötigt. Es wird zudem noch ein Mechanismus benötigt der die Bälle in den Korb befördert.
Abschliessend wird noch das Endsignal übermittelt.
Damit das Gerät seine Aufgabe erfüllen kann, benötigt es noch eine Energieversorgung.
Somit ergeben sich folgende Kernfunktionen:

\begin{itemize}
    \item Anpeilen des Korbes
    \item Ballbeförderung
    \item Korberkennung
    \item Balllager
    \item Drahtlose Übermittlung des Startsignals
    \item Energieversorgung
\end{itemize}

\subsection{Anpeilen des Korbes}
\begin{itemize}
    \item Fliegend \\
        Das Gerät fliegt zum Korb. 
    \item Fahrend \\
        Das Gerät wird fahrend vor den Korb verschoben. 
    \item Stehend \\
        Das Gerät wird um die eigene Achse rotierend zum Korb ausgerichtet. 
\end{itemize}

\subsection{Ballbeförderung}
\begin{itemize}
    \item Ballistisch \\
        Der Ball wird geworfen und fliegt auf einer ballistischen Flugbahn zum 
        Korb. 
    \item Abwurf \\
        Der Ball wird von oberhalb des Korbes abgeworfen. 
\end{itemize}

\subsection{Korberkennung}
\begin{itemize}
    \item Distanzmessung \\
        Die Distanz zur Wand wird gemessen. Dabei wird die gesamte Breite 
        gescant. An der Stelle mit der geringsten Distanz ist der Korb 
        positioniert. 
    \item Optische Erkennung \\
        Das Spielfeld wird mit einer Kamera aufgezeichnet. Mittels 
        Bildverarbeitung wird der Korb erkannt und dessen Position ermittelt. 
\end{itemize}

\subsection{Balllager}
\begin{itemize}
    \item Einzeln \\
        Die Bälle werden einzeln gelagert und befördert. 
    \item Zusammen \\
        Die Bälle werden als eine Einheit gelagert und befördert. 
\end{itemize}

\subsection{Drahtlose Übermittlung des Startsignals}
\begin{itemize}
    \item Funk \\
        Die Kommunikation erfolgt via elektromagnetischer Wellen. 
    \item Optisch \\
        Die Kommunikation erfolgt via Licht. 
    \item Ultraschall \\
        Die Kommunikation erfolgt via Schallwellen im Ultraschallbereich 
        (> 20 kHz). 
    \item Spracherkennung \\
        Das Startsignal wird von einer Person ausgesprochen und vom Gerät 
        mittels Spracherkennung decodiert. 
\end{itemize}

\subsection{Energieversorgung}
\begin{itemize}
    \item intern elektrisch \\
        Das Gerät wird durch einen elektrochemischen Energiespeicher 
        angetrieben. 
    \item extern elektrisch \\
        Das Gerät wird von einem Speisegerät mit Energie versorgt. 
    \item Verbrennungsmotor \\
        Das Gerät wird von einem Verbrennungsmotor angetrieben. 
    \item Druckluft \\
        Die Bälle werden pneumatisch befördert. 
    \item Potenzielle Energie \\
        Die Bälle werden nur mittels der in ihnen gespeicherten Energie 
        befördert. 
    \item Feder \\
        Die Bälle werden von einer Feder beschleunigt. 
\end{itemize}

