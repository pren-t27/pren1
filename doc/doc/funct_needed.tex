\section{Benötigte Funktionen}
Aus der Aufgabenstellung können die für das Lösen der Aufgabe benötigten 
Funktionen extrahiert werden.  Diese bestehen aus einem Balllager welches die 
Tennisbälle vor dem Start aufnimmt. Des weiteren wird, um das Gerät arbeiten 
zu lassen, ein Startsignal benötigt, welches drahtlos übertragen werden soll. 
Weitere Teilfunktionen sind die Art der Fortbewegung und die Korberkennung. Je 
nach Art der Fortbewegung wird eine andere Erkennung benötigt. Es wird zudem 
ein Mechanismus benötigt, der die Bälle in den Korb befördert.  Abschliessend 
wird das Endsignal übermittelt.  Damit das Gerät seine Aufgabe erfüllen kann, 
benötigt es noch eine Energieversorgung.  Somit ergeben sich folgende 
Teilfunktionen:

\begin{itemize}
    \item Art der Bewegung
    \item Ballbeförderung
    \item Korberkennung
    \item Balllager
    \item Übermittlung Start- / Endsignal
    \item Energieversorgung
\end{itemize}
Für jede dieser Teilfunktionen werden nun verschiedene Lösungsvarianten 
erarbeitet. Dies ergibt jeweils eine Zeile des morphologischen Kastens. 

\subsection{Art der Bewegung}
Das Gerät muss zum Korb ausgerichtet werden. Wenn ein fliegendes Gerät gebaut 
wird, wird das ganze Gerät oder ein Teil davon zum Korb geflogen. Wenn das 
Gerät fahren kann, wird es fahrend vor den Korb verschoben. Wenn das Gerät an 
einem festen Standort steht, wird es um seine eigene Achse drehend zum Korb 
ausgerichtet. 
% Morphologischer Kasten mit Stichworten
\footnotesize
\begin{table}[h!]
    \centering
    \begin{zebratabular}{@{}p{0.23\linewidth}p{\morphcellwidth}p{\morphcellwidth}p{\morphcellwidth}p{\morphcellwidth}p{\morphcellwidth}p{\morphcellwidth}}
    %\begin{zebratabular}{@{}p{0.23\linewidth}llllll}
        \rowcolor{gray}
        Eigenschaften &
            \multicolumn{6}{l}{Merkmalausprägung} \\
        Art der Bewegung &
            Fliegend                     &
            Fahrend                      &
            Stehend                      &
                                         &
                                         &
                                         \\
    \end{zebratabular}
    % Linien zur Darstellung von Lösungsvarianten
    \caption{Morphologischer Kasten Bereich für Art der Bewegung}
\end{table}
\normalsize

\subsection{Ballbeförderung}
Die Beförderung der Bälle ist abhängig von der Art der Bewegung des Gerätes. 
Wenn das Gerät am Boden verbleibt, werden die Bälle geworfen und folgen einer 
ballistischen Flugbahn. Bei einem Fluggerät werden die Bälle von oberhalb des 
Korbes abgeworfen. 
% Morphologischer Kasten mit Stichworten
\footnotesize
\begin{table}[h!]
    \centering
    \begin{zebratabular}{@{}p{0.23\linewidth}p{\morphcellwidth}p{\morphcellwidth}p{\morphcellwidth}p{\morphcellwidth}p{\morphcellwidth}p{\morphcellwidth}}
    %\begin{zebratabular}{@{}p{0.23\linewidth}llllll}
        \rowcolor{gray}
        Eigenschaften &
            \multicolumn{6}{l}{Merkmalausprägung} \\
        Ballbeförderung &
            Ballistisch                  &
            Abwurf                       &
                                         &
                                         &
                                         &
                                         \\
    \end{zebratabular}
    % Linien zur Darstellung von Lösungsvarianten
    \caption{Morphologischer Kasten Bereich für Ballbeförderung}
\end{table}
\normalsize

\subsection{Korberkennung}
Für die Korberkennung kann ein Sensor zur Distanzmessung verwendet werden. 
Mit diesem kann das Spielfeld gescannt werden. In der Richtung, in welche die 
kleinste Distanz gemessen wird, befindet sich der Korb. Das Spielfeld kann 
auch mit einer Kamera aufgenommen werden. Mittels Bildverarbeitung wird der 
Korb erkannt und dessen Position ermittelt. 
% Morphologischer Kasten mit Stichworten
\footnotesize
\begin{table}[h!]
    \centering
    \begin{zebratabular}{@{}p{0.23\linewidth}p{\morphcellwidth}p{\morphcellwidth}p{\morphcellwidth}p{\morphcellwidth}p{\morphcellwidth}p{\morphcellwidth}}
    %\begin{zebratabular}{@{}p{0.23\linewidth}llllll}
        \rowcolor{gray}
        Eigenschaften &
            \multicolumn{6}{l}{Merkmalausprägung} \\
        Korberkennung &
            Distanzmessung               &
            Optische Erkennung           &
                                         &
                                         &
                                         &
                                         \\
    \end{zebratabular}
    % Linien zur Darstellung von Lösungsvarianten
    \caption{Morphologischer Kasten Bereich für Korberkennung}
\end{table}
\normalsize

\subsection{Balllager}
Das Gerät muss die Bälle vor dem Start aufnehmen. Die Bälle können einzeln 
gelagert und befördert werden. Alternativ können die Bälle als eine Einheit 
gelagert und beförtert werden. 
% Morphologischer Kasten mit Stichworten
\footnotesize
\begin{table}[h!]
    \centering
    \begin{zebratabular}{@{}p{0.23\linewidth}p{\morphcellwidth}p{\morphcellwidth}p{\morphcellwidth}p{\morphcellwidth}p{\morphcellwidth}p{\morphcellwidth}}
    %\begin{zebratabular}{@{}p{0.23\linewidth}llllll}
        \rowcolor{gray}
        Eigenschaften &
            \multicolumn{6}{l}{Merkmalausprägung} \\
        Balllager &
            Einzeln                      &
            Zusammen                     &
                                         &
                                         &
                                         &
                                         \\
    \end{zebratabular}
    % Linien zur Darstellung von Lösungsvarianten
    \caption{Morphologischer Kasten Bereich für Balllager}
\end{table}
\normalsize

\subsection{Übermittlung Start- / Endsignal}
Da das Startsignal drahtlos übertragen werden muss, kann dafür kein Kabel 
verwendet werden. Naheliegend ist die Verwendung von elektromagnetischen 
Wellen (Funk). Ausserdem kann das Signal optisch mittels Lichtwellen 
übertragen werden. Die Kommunikation kann auch mit Schallwellen im nicht 
hörbaren Bereich (Ultraschall) erfolgen. Ist das Gerät mit einer 
Spracherkennung ausgerüstet, kann das Gerät gesprochene Befehle interpretieren. 
% Morphologischer Kasten mit Stichworten
\footnotesize
\begin{table}[h!]
    \centering
    \begin{zebratabular}{@{}p{0.23\linewidth}p{\morphcellwidth}p{\morphcellwidth}p{\morphcellwidth}p{\morphcellwidth}p{\morphcellwidth}p{\morphcellwidth}}
    %\begin{zebratabular}{@{}p{0.23\linewidth}llllll}
        \rowcolor{gray}
        Eigenschaften &
            \multicolumn{6}{l}{Merkmalausprägung} \\
        Übermittlung Start- / Endsignal &
            Funk                         &
            Optisch                      &
            Ultraschall                  &
            Sprach\-erkennung            &
                                         &
                                         \\
    \end{zebratabular}
    % Linien zur Darstellung von Lösungsvarianten
    \caption{Morphologischer Kasten Bereich für Übermittlung Start- / Endsignal}
\end{table}
\normalsize

\subsection{Energieversorgung}
Das Gerät kann mit einem internen elektrischen Energiespeicher ausgerüstet 
sein. Die elektrische Energie kann jedoch auch von extern mit einem 
Speisegerät zur Verfügung gestellt werden. Die Bälle können auch mit Hilfe 
eines Verbrennungsmotors angetrieben werden. Auch Druckluft kann für den 
Antrieb des Gerätes dienen. Die Bälle können aber auch nur mithilfe von 
potenzieller Energie beschleunigt werden, indem sie zum Beispiel eine Rampe 
hinunterrollen. Die Bälle können auch mit einer Feder beschleunigt werden. 
% Morphologischer Kasten mit Stichworten
\footnotesize
\begin{table}[h!]
    \centering
    \begin{zebratabular}{@{}p{0.23\linewidth}p{\morphcellwidth}p{\morphcellwidth}p{\morphcellwidth}p{\morphcellwidth}p{\morphcellwidth}p{\morphcellwidth}}
    %\begin{zebratabular}{@{}p{0.23\linewidth}llllll}
        \rowcolor{gray}
        Eigenschaften &
            \multicolumn{6}{l}{Merkmalausprägung} \\
        Energieversorgung &
            intern elektrisch            &
            extern elektrisch            &
            Ver\-bren\-nungs\-mo\-tor    &
            Druckluft                    &
            Potenzielle Energie          &
            Feder                        \\
    \end{zebratabular}
    % Linien zur Darstellung von Lösungsvarianten
    \caption{Morphologischer Kasten Bereich für Energieversorgung}
\end{table}
\normalsize

