\section{Benötigte Funktionen}
Aus der Aufgabenstellung können die für das Lösen der Aufgabe benötigten 
Funktionen extrahiert werden.  Diese bestehen aus einem Balllager welches die 
Tennisbälle vor dem Start aufnimmt. Des weiteren wird, um das Gerät arbeiten 
zu lassen, ein Startsignal benötigt, welches drahtlos übertragen werden soll. 
Weitere Teilfunktionen sind die Art der Fortbewegung und die Korberkennung. Je 
nach Art der Fortbewegung wird eine andere Erkennung benötigt. Es wird zudem 
ein Mechanismus benötigt, der die Bälle in den Korb befördert.  Abschliessend 
wird das Endsignal übermittelt.  Damit das Gerät seine Aufgabe erfüllen kann, 
benötigt es noch eine Energieversorgung.  Somit ergeben sich folgende 
Teilfunktionen:

\begin{itemize}
    \item Art der Bewegung
    \item Ballbeförderung
    \item Korberkennung
    \item Balllager
    \item Übermittlung Start- / Endsignal
    \item Energieversorgung
\end{itemize}

\subsection{Art der Bewegung}
\begin{itemize}
    \item Fliegend \\
        Das Gerät fliegt zum Korb. 
    \item Fahrend \\
        Das Gerät wird fahrend vor den Korb verschoben. 
    \item Stehend \\
        Das Gerät wird um die eigene Achse rotierend zum Korb ausgerichtet. 
\end{itemize}
% Morphologischer Kasten mit Stichworten
\footnotesize
\begin{table}[h!]
    \centering
    \begin{zebratabular}{@{}p{0.23\linewidth}p{\morphcellwidth}p{\morphcellwidth}p{\morphcellwidth}p{\morphcellwidth}p{\morphcellwidth}p{\morphcellwidth}}
    %\begin{zebratabular}{@{}p{0.23\linewidth}llllll}
        \rowcolor{gray}
        Eigenschaften &
            \multicolumn{6}{l}{Merkmalausprägung} \\
        Art der Bewegung &
            Fliegend                     &
            Fahrend                      &
            Stehend                      &
                                         &
                                         &
                                         \\
    \end{zebratabular}
    % Linien zur Darstellung von Lösungsvarianten
    \caption{Morphologischer Kasten}
\end{table}
\normalsize


\subsection{Ballbeförderung}
\begin{itemize}
    \item Ballistisch \\
        Der Ball wird geworfen und fliegt auf einer ballistischen Flugbahn zum 
        Korb. 
    \item Abwurf \\
        Der Ball wird von oberhalb des Korbes abgeworfen. 
\end{itemize}
% Morphologischer Kasten mit Stichworten
\footnotesize
\begin{table}[h!]
    \centering
    \begin{zebratabular}{@{}p{0.23\linewidth}p{\morphcellwidth}p{\morphcellwidth}p{\morphcellwidth}p{\morphcellwidth}p{\morphcellwidth}p{\morphcellwidth}}
    %\begin{zebratabular}{@{}p{0.23\linewidth}llllll}
        \rowcolor{gray}
        Eigenschaften &
            \multicolumn{6}{l}{Merkmalausprägung} \\
        Ballbeförderung &
            Ballistisch                  &
            Abwurf                       &
                                         &
                                         &
                                         &
                                         \\
    \end{zebratabular}
    % Linien zur Darstellung von Lösungsvarianten
    \caption{Morphologischer Kasten}
\end{table}
\normalsize


\subsection{Korberkennung}
\begin{itemize}
    \item Distanzmessung \\
        Die Distanz zur Wand wird gemessen. Dabei wird die gesamte Breite 
        gescant. An der Stelle mit der geringsten Distanz ist der Korb 
        positioniert. 
    \item Optische Erkennung \\
        Das Spielfeld wird mit einer Kamera aufgezeichnet. Mittels 
        Bildverarbeitung wird der Korb erkannt und dessen Position ermittelt. 
\end{itemize}
% Morphologischer Kasten mit Stichworten
\footnotesize
\begin{table}[h!]
    \centering
    \begin{zebratabular}{@{}p{0.23\linewidth}p{\morphcellwidth}p{\morphcellwidth}p{\morphcellwidth}p{\morphcellwidth}p{\morphcellwidth}p{\morphcellwidth}}
    %\begin{zebratabular}{@{}p{0.23\linewidth}llllll}
        \rowcolor{gray}
        Eigenschaften &
            \multicolumn{6}{l}{Merkmalausprägung} \\
        Korberkennung &
            Distanzmessung               &
            Optische Erkennung           &
                                         &
                                         &
                                         &
                                         \\
    \end{zebratabular}
    % Linien zur Darstellung von Lösungsvarianten
    \caption{Morphologischer Kasten}
\end{table}
\normalsize


\subsection{Balllager}
\begin{itemize}
    \item Einzeln \\
        Die Bälle werden einzeln gelagert und befördert. 
    \item Zusammen \\
        Die Bälle werden als eine Einheit gelagert und befördert. 
\end{itemize}
% Morphologischer Kasten mit Stichworten
\footnotesize
\begin{table}[h!]
    \centering
    \begin{zebratabular}{@{}p{0.23\linewidth}p{\morphcellwidth}p{\morphcellwidth}p{\morphcellwidth}p{\morphcellwidth}p{\morphcellwidth}p{\morphcellwidth}}
    %\begin{zebratabular}{@{}p{0.23\linewidth}llllll}
        \rowcolor{gray}
        Eigenschaften &
            \multicolumn{6}{l}{Merkmalausprägung} \\
        Balllager &
            Einzeln                      &
            Zusammen                     &
                                         &
                                         &
                                         &
                                         \\
    \end{zebratabular}
    % Linien zur Darstellung von Lösungsvarianten
    \caption{Morphologischer Kasten}
\end{table}
\normalsize


\subsection{Übermittlung Start- / Endsignal}
\begin{itemize}
    \item Funk \\
        Die Kommunikation erfolgt via elektromagnetischer Wellen. 
    \item Optisch \\
        Die Kommunikation erfolgt via Licht. 
    \item Ultraschall \\
        Die Kommunikation erfolgt via Schallwellen im Ultraschallbereich 
        (> 20 kHz). 
    \item Spracherkennung \\
        Das Startsignal wird von einer Person ausgesprochen und vom Gerät 
        mittels Spracherkennung decodiert. 
\end{itemize}
% Morphologischer Kasten mit Stichworten
\footnotesize
\begin{table}[h!]
    \centering
    \begin{zebratabular}{@{}p{0.23\linewidth}p{\morphcellwidth}p{\morphcellwidth}p{\morphcellwidth}p{\morphcellwidth}p{\morphcellwidth}p{\morphcellwidth}}
    %\begin{zebratabular}{@{}p{0.23\linewidth}llllll}
        \rowcolor{gray}
        Eigenschaften &
            \multicolumn{6}{l}{Merkmalausprägung} \\
        Übermittlung Start- / Endsignal &
            Funk                         &
            Optisch                      &
            Ultraschall                  &
            Sprach\-erkennung            &
                                         &
                                         \\
    \end{zebratabular}
    % Linien zur Darstellung von Lösungsvarianten
    \caption{Morphologischer Kasten}
\end{table}
\normalsize


\subsection{Energieversorgung}
\begin{itemize}
    \item intern elektrisch \\
        Das Gerät wird durch einen elektrochemischen Energiespeicher 
        angetrieben. 
    \item extern elektrisch \\
        Das Gerät wird von einem Speisegerät mit Energie versorgt. 
    \item Verbrennungsmotor \\
        Das Gerät wird von einem Verbrennungsmotor angetrieben. 
    \item Druckluft \\
        Die Bälle werden pneumatisch befördert. 
    \item Potenzielle Energie \\
        Die Bälle werden nur mittels der in ihnen gespeicherten Energie 
        befördert. 
    \item Feder \\
        Die Bälle werden von einer Feder beschleunigt. 
\end{itemize}
% Morphologischer Kasten mit Stichworten
\footnotesize
\begin{table}[h!]
    \centering
    \begin{zebratabular}{@{}p{0.23\linewidth}p{\morphcellwidth}p{\morphcellwidth}p{\morphcellwidth}p{\morphcellwidth}p{\morphcellwidth}p{\morphcellwidth}}
    %\begin{zebratabular}{@{}p{0.23\linewidth}llllll}
        \rowcolor{gray}
        Eigenschaften &
            \multicolumn{6}{l}{Merkmalausprägung} \\
        Energieversorgung &
            intern elektrisch            &
            extern elektrisch            &
            Ver\-bren\-nungs\-mo\-tor    &
            Druckluft                    &
            Potenzielle Energie          &
            Feder                        \\
    \end{zebratabular}
    % Linien zur Darstellung von Lösungsvarianten
    \caption{Morphologischer Kasten}
\end{table}
\normalsize


