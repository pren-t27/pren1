\documentclass[a4paper,10pt,fleqn]{article}

\usepackage{../layout/layout}

\begin{document}

\section*{Sitzungsprotokoll Nr. 7}
Modul PREN1, Gruppe 27

\begin{longtable}[l]{@{}p{0.25\textwidth}@{}p{0.75\textwidth}@{}}
    Datum / Zeit: &
        13.11.2014 8:30
        \\\\
    Ort: &
        HSLU, Horw
        \\\\
    Vorsitz: &
        PK
        \\\\
    Anwesend: &
        Yannik Küng (YK) ~
        Andriu Maissen (AM) ~
        Daniel Mathis (DM) ~
        Simon Neidhart (SN) ~
        Peter Kuonen (PK) ~
        Kevin Wespi (KW) ~
        Daniel Winz (DW) ~
        % - ~
        \\\\
    Entschuldigt: &
        % Yannik Küng (YK) ~
        % Andriu Maissen (AM) ~
        % Daniel Mathis (DM) ~
        % Simon Neidhart (SN) ~
        % Peter Kuonen (PK) ~
        % Kevin Wespi (KW) ~
        % Daniel Winz (DW) ~
        - ~
        \\\\
    Gäste: &
        % Marco De Angelis
        - ~
        \\\\
    Protokollführer: &
        KW
        \\
\end{longtable}
%
\subsection*{Beschlüsse, Erkenntnisse, Aufgaben, Pendenzen:}
\begin{itemize}
    \item Besprechung des letzten Meilensteins mit Dozent
    \begin{itemize}
    	\item Nachvollziehbarkeit, Entscheidungen müssen begründet sein
    	\begin{itemize}
    		\item in Tabellen nicht genau ersichtlich, wie wurde entschieden
    		\item wie ist Punkteverteilung erfolgt?
    		\item aussagekräftige Texte bei den Tabellen
    	\end{itemize}
    	\item Risikobeurteilung sehr theoretisch
    	\item Reihenfolge der Objekte anpassen 
    	\item Gegenüberstellung der Objekte, wie ist Punkteverteilung erfolgt im Anhang ergänzen
    	\item um Rechnung zu komplettieren, sollten Winkelfehler und Startgeschwindigkeitsfehler berechnet werden und eine Schlussfolgerung gezogen werden
    \end{itemize}
    \item ToDo's für nächsten Meilenstein
    \begin{itemize}
    	\item siehe Excel-Liste in Dropbox
    \end{itemize}
    \item Verteilung der neuen Aufgaben
\end{itemize}
%
\subsection*{Traktanden für die nächste Sitzung}
\begin{itemize}
    \item Individuelle Arbeit an den neuen Aufgaben
    \begin{itemize}
    	\item
	\end{itemize}
    
\end{itemize}
%
\end{document}
