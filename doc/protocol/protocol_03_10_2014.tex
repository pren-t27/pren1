\documentclass[a4paper,10pt,fleqn]{article}

\usepackage{../layout/layout}

\begin{document}

\section*{Sitzungsprotokoll Nr. 1}
Modul PREN1, Gruppe 27

\begin{longtable}[l]{@{}p{0.25\textwidth}@{}p{0.75\textwidth}@{}}
    Datum / Zeit: &
        26.09.2014 10:00
        \\\\
    Ort: &
        HSLU, Horw
        \\\\
    Vorsitz: &
        PK
        \\\\
    Anwesend: &
        Yannik Küng (YK) ~
        Andriu Maissen (AM) ~
        Daniel Mathis (DM) ~
        Simon Neidhart (SN) ~
        Peter Kuonen (PK) ~
        Kevin Wespi (KW) ~
        Daniel Winz (DW) ~
        % - ~
        \\\\
    Entschuldigt: &
        % Yannik Küng (YK) ~
        % Andriu Maissen (AM) ~
        % Daniel Mathis (DM) ~
        % Simon Neidhart (SN) ~
        % Peter Kuonen (PK) ~
        % Kevin Wespi (KW) ~
        % Daniel Winz (DW) ~
        - ~
        \\\\
    Gäste: &
        % Marco De Angelis
        - ~
        \\\\
    Protokollführer: &
        KW
        \\
\end{longtable}
%
\subsection*{Beschlüsse, Erkenntnisse, Aufgaben, Pendenzen:}
\begin{itemize}
•	Besprechung der morphologischen Kästen
•	Feedback DeAngelis zum Testat
o	Projektanforderungen:
	Lieber kleine Sätze als Aufzählungen
	Fehler in Grössen, Bezeichnungen (zB. Mittellinie)
	Notschalter genauer erläutern
	Komplexe Versionsnummer, Version sollte ersichtlich sein, Datum
	Gute eigene Ziele, Einschränkungen
	Entweder/oder bei Flugobjekt, Bodenobjekt klarstellen, evtl. separieren
	Zeichnungen ergänzen (auch Copy-Paste aus Projektauftrag)
	Messbarkeit der Resultate, Beweisbarkeit
o	Technologierecherche:
	Bewertung nach Nutzern fraglich, besser allgemein bewerten, Zuverlässigkeit der Quellen
	Zum Teil Lösungen schon zu detailliert bewertet
•	Was wäre die bessere Lösung, allgemein
•	Nicht Technologie, sondern Lösung bewerten
•	Quellen bewerten, Umfang und Zuverlässigkeit
•	Höheres Niveau der Abstraktion
•	Voreilig in der Bewertung der Lösungen
•	Technologierecherche passen wir an

\end{itemize}
%
\subsection*{Traktanden für die nächste Sitzung}
\begin{itemize}
    \item Technologierecherche
    
\end{itemize}
%
\end{document}
