\documentclass[a4paper,10pt,fleqn]{article}

\usepackage{../layout/layout}

\begin{document}

\section*{Sitzungsprotokoll Nr. 5}
Modul PREN1, Gruppe 27

\begin{longtable}[l]{@{}p{0.25\textwidth}@{}p{0.75\textwidth}@{}}
    Datum / Zeit: &
        30.10.2014 8:30
        \\\\
    Ort: &
        HSLU, Horw
        \\\\
    Vorsitz: &
        PK
        \\\\
    Anwesend: &
        Yannik Küng (YK) ~
        Andriu Maissen (AM) ~
        Daniel Mathis (DM) ~
        Simon Neidhart (SN) ~
        Peter Kuonen (PK) ~
        Kevin Wespi (KW) ~
        Daniel Winz (DW) ~
        % - ~
        \\\\
    Entschuldigt: &
        % Yannik Küng (YK) ~
        % Andriu Maissen (AM) ~
        % Daniel Mathis (DM) ~
        % Simon Neidhart (SN) ~
        % Peter Kuonen (PK) ~
        % Kevin Wespi (KW) ~
        % Daniel Winz (DW) ~
        - ~
        \\\\
    Gäste: &
        % Marco De Angelis
        - ~
        \\\\
    Protokollführer: &
        KW
        \\
\end{longtable}
%
\subsection*{Beschlüsse, Erkenntnisse, Aufgaben, Pendenzen:}
\begin{itemize}
    \item Teamcoaching mit Dozententeam
    \begin{itemize}
        \item PK stellt aktuellen Stand vor
        \item statischer Turm
        \item ist Geschwindigkeit und Umdrehungen geprüft?
        \item Warum quadratische Zuführung
        \item Ziel unter 2 kg
        \item Trägheit des Gesamtsystems
    \end{itemize}
\end{itemize}
%
\subsection*{Traktanden für die nächste Sitzung}
\begin{itemize}
    \item Morphologische Kästen auswerten
    \begin{itemize}
    	\item KW: fliegendes Objekt
    	\item YK: fahrendes Objekt
    	\item AM: stehendes Objekt
	\end{itemize}
    \item Aufbau:
	    \begin{itemize}
		    \item Kurzbeschrieb
		    \item Bild morphologischer Kasten
		    \item Beschreibung morphologischer Kasten
		    \item Kriterien Skala 1-10
		    \begin{itemize}
		    	\item Gewicht
		    	\item Zeit
		    	\item Genauigkeit
		    	\item Risikofaktor
		    	\item Aufwand
		    \end{itemize}
	    \end{itemize}
    
\end{itemize}
%
\end{document}
