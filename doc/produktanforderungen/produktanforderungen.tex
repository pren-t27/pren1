\documentclass[a4paper,10pt,fleqn]{article}

\usepackage{../layout/layout}

\title{Produktanforderungen}
\author{Gruppe 27}

\begin{document}

% \maketitle
% \clearpage
% \tableofcontents
% \clearpage

\section{Produktanforderungen}

\subsection{Ziel}
\begin{itemize}
    \item möglichst viele der fünf Tennisbälle in kurzer Zeit in den Korb
    \item Aufgabe autonom bewältigen, keine Eingriffe nach Start
    \item Korb muss selbständig gefunden werden
    \item möglichst leicht (Akku, Speisegerät, Stationärer Rechner werden 
        nicht mitgerechnet)
\end{itemize}

\subsection{Spielfeld}
\begin{itemize}
    \item bis zur Begrenzungslinie darf befahren, beschritten, bekrochen werden
    \item zwischen Begrenzungslinie und Korb darf nur überworfen oder überflogen
    \item darf nicht verändert werden
    \item Spielfeldrand nicht umgreifen
\end{itemize}

\subsection{Gerät}
\begin{itemize}
    \item muss eine Eigenkonstruktion sein
    \item Systemkomponenten wie z.B. Servos, das Lenkgetriebe, usw. können 
        gekauft und eingesetzt werden
    \item Startbefehl muss drahtlos übermittelt werden
    \item Kommunikationsgerät muss optisch oder akustisch Endsignal anzeigen
    \item möglichst einfach ohne diese Komponenten (z.B. Akku) wägen lassen
    \item Aufhängevorrichtung muss zum Wägen vorhanden sein
    \item \textbf{maximalen Abmessungen: 0.5 m x 0.5 m x 1 m (Auftriebskörper 
        nicht mitgerechnet)}
    \item muss in einem Hindernisfreien Raum von der Höhe 1.8m und einem 
        Abstand von 0.5m um das Feld operieren können
\end{itemize}

\subsection{Zusätzlich}
\begin{itemize}
    \item Das Gerät muss möglichst viele der fünf Tennisbälle, die Sie 
        vorgängig erhalten, in möglichst kurzer Zeit in einen Korb befördern. 
    \item Das Spielfeld bis zur Begrenzungslinie darf befahren, beschritten, 
        bekrochen, überflogen, überragt und auch überworfen werden.
    \item Das Feld zwischen Begrenzungslinie und Korb darf nur überworfen oder 
        überflogen werden
    \item Das Spielfeld darf nicht verändert werden.
    \item Das System (Gerät, Steuerung, Kommunikation…) muss eine 
        Eigenkonstruktion sein. Einzelne Systemkomponenten wie z.B. Servos, 
        das Lenkgetriebe eines Modellautos, ein Sendemodul oder eine Kamera 
        dürfen zugekauft und eingesetzt werden.  
    \item Das Gerät muss die Aufgabe autonom bewältigen. Nach dem Startbefehl 
        dürfen keine Eingriffe mehr vorgenommen werden. Insbesondere muss das 
        Gerät die Position des Korbes selbständig finden.  
    \item Der Startbefehl muss drahtlos von einem Smartphone, Tablet, PC oder 
        Laptop aus übermittelt 
        werden. 
    \item Auf dem gleichen Kommunikationsgerät muss optisch oder akustisch 
        angezeigt werden, wann die Aufgabe abgeschlossen ist und die Zeit 
        gestoppt werden kann. 
    \item Das Gerät darf den Spielfeldrand nicht umgreifen. 
    \item Das Gerät soll möglichst leicht sein. Die Energieversorgung (Akku, 
        Speisegerät, Druckluft-versorgung…), das zum Starten des Geräts 
        verwendete Kommunikationsgerät sowie ein allenfalls eingesetzter 
        zusätzlicher stationärer Rechner werden nicht mitgerechnet. 
    \item Das Gerät soll sich möglichst einfach ohne diese Komponenten (z.B. 
        Akku) wägen lassen. 
    \item Zum Wägen ist am Gerät eine Aufhängevorrichtung vorzusehen, damit 
        das Gerät mit einer Federwage gewogen werden kann. 
    \item Die maximalen Abmessungen des Gerätes – auch während des Ausführens 
        der Aufgabe – betragen 0.5 m x 0.5 m x 1 m. Ein allfällig zusätzlich 
        eingesetzter Rechner fällt nicht unter diese Grössen-beschränkung. 
        Falls das Gerät fliegt, fällt ein Auftriebskörper nicht unter diese 
        Grössenbeschränkung. Auch dürfen Rotoren oder Flüge über das Mass 
        hinausragen. 
    \item Die Personensicherheit muss jederzeit gewährleistet sein. 
    \item Das Gerät muss in einem Hindernisfreien Raum von der Höhe 1.8m und 
        einem Abstand von 0.5m um das Feld operieren können.
    \item Das System muss innerhalb von 5 Minuten startklar gemacht werden.
    \item Für den Bau der Teilfunktionsmuster und für die Realisierung des 
        Systems stehen insgesamt CHF 600.- zur Verfügung. Davon dürfen maximal 
        CHF 200.- für das Teilfunktionsmuster ausgegeben werden.  Die Kosten 
        für Normteile wie Schrauben, Lager, Rohmaterial, Widerstände, 
        Kondensatoren usw. werden nicht verrechnet, sofern die Teile gemäss 
        Lagerliste in den Werkstätten der HSLU - T\&A am Lager sind.
\end{itemize}

\end{document}
